\documentclass{article}
\usepackage[utf8]{inputenc}


\begin{document}
\subsection{Preliminär plan:}

1 titel + undertitel | + en mening

	Metod för komposition 
	- Minnesbilder som metod för komponerande.

	

2 längre beskrivning, ungefär en a4, förklarar vad som arbetet ska innehålla

Jag vill utforska hur en strategi som baseras på att skriva "beskrivande" musik, där musiken på ett eller
annat sätt influeras eller ska gestalta en känsla eller minne, kan ge för effekt på generativ, algoritmisk och
i övrigt syntetisk musik, inom vilken genre man annars ofta lägger stort fokus på hur tekniken bakom är
utformad. Om man kan skilja på ett beskrivand "subjektivt" perspektivt, och ett mer tekniskt "objektivt"
perspektivt, och om de gynnas av varann?

Jag har upplevt att jag länge försökt skriva musik utifrån en ny teknik jag lärt mig. Att det är den som har
det verkliga värdet, den tekniska aspekten av musiken. Jag har också ofta haft det som startposition, att mitt
komponerande varit ett utforskande av verktyg, snarare än utforskandet av själva musiken som det kan
frambringa. Att jag, i en kultur som gärna intresserar sig av teknologi, fallit i fällan att tekniken har ett
egenvärde. Även ett tillräckligt stort värde för att kunna bära den musiken jag skriver. Att det för mig var
lätt att då ha svårt med frågeställningar som "vad vill musiken säga?" mer än att det då blir en "tech demo".


Terapeutiskt komponerande? Att utgå ifrån en händelse, som ett ramverk att utgå ifrån när konstnärliga val ska
tas. Använda sig av ett figurativt perspektiv för att berätta en historia.

Jag har, som ett projekt under min sjukskrivning försökt dokumentera den genom att skriva om den, som i en
dagbok. Eftersom det har varit en rätt märklig och ovan upplevelse, som också har ändrats under tidens gång,
med nytt perspektiv i retrospekt, har jag kokat ner de här anteckningarna till någon form av snapshots,
generaliserade bilder av för mig tydligt avdelningsbara perioder. När jag väl valt ut bilder försökte jag
kategorisera för mig själv hur jag ville att dessa bilders olika känslor skulle låta, om det handlade om något
introspektivt eller om miljön och tiden. Tillsammans med en kort sammanfattning skrev jag även ner idéer om
textur och gestik som jag associerar till den bilden, och detta blev sedan någon slags karta för ett antal
kompositioner med ett sammanhängande tema, en byggställning för att bygga runt och några första idéer. Det
gjorde att jag hade något väldigt konkret att arbeta med, men som jag hade en personlig och känslomässig
koppling till. För mig blev det att jag inte ville lämna projektet för fort, det fanns en motivation att
fortsätta och fullborda arbetet, även om det inte alls blir som det var tänkt (citat den där examensuppsatsen
jag läste).

Det som Niklas sa om Linus examensarbete, att det är intressant att delta i ny "emerging" teknologi, som inte
än har fixerats i nåt normaliserat. Ett uttryck som fortfarande utforskas. Diskuterar han kring om teknologi
hämmar eller bara lyfter kreativitet? 

Kan teknologin vara något romantiskt? 

Poietisk process		Neutral			   Estesisk process
Minneanteckning ("ursprung") -> tolkning (producent) -> stycke (medium) -> tolkning (mottagare)
	Vad för form erhålls från musik 2 steg bortkopplade från objektet den gestaltar? Spelar det någon roll?
	Vad är kvar? 

Utesluter de två perspektiven varann, eller kompletterar de varann?

Kan det vara värt att utforska Mark Fisher och det han skriver om skolsystem och undervisning under i ett
kapitalistist samhälle, när han pratar om apati som drabbar studenter i ett system som ibland curlar dem, där
skolor behöver godkända elever för att få anslag. Har det nåt med nåt att göra eller tycker jag bara det är
intressant? 

Den elektroakustiska musiken som fält har från sin början motiverat sin existens genom vetenskaplig forskning. 
Det finns därför en tydlig förankring mellan musiken och teknik och teknologisk utveckling. Detta har banat 
väg för högre studier inom elektroakustisk musik.

Hur har det bidragit till att forma elektroakustisk musik? På vilka sätt har det bidragit eller begränsat ett
uttryck eller den sociala läromiljö som högre studier inom komposition innebär? 

Att bli bedömd utifrån ens verktyg, t ex använda logic istället för något med bökigare inlärningskurva (läs
SuperCollider eller något liknande), kan ses som ett tecken på professionalitet (som visitkorten i American
Psycho).

Vad är begreppet DIY? 
Det dikterar inte att ett hemmabyggt, hemmakodat reverb är bättre än det som kommer färdigt med din DAW. Det
handlar mer om en estetik som har en grund i koncept som självlärande och ett gemensamt utforskande (som
studiecirklar) och deltagande som kan beskrivas som voluntärt, mer än ett strängt förhållningssätt till
slagorden "gör det själv".

Ibland kan studier i EAM bli ett utvecklande av ens stora allomfattande Instrument™. I utvecklandet ens
Instrument™ har man lärt sig mycket, men syftet kan ha försvunnit på vägen, i form av den musiken du
föreställde dig att du skulle skapa, bara Instrumentet™ var färdigbyggt.

Utforska ett figurativt berättande. Brodera ut kring hur man talar kring eam, varför samtal om tekniska
aspekter ofta premieras över samtal om en mer emotionell eller poetisk, empatisk eller figurativ grund i
musiken. 


\subsection{Referenser:}

Brian Eno - Music For Airports - Audio Culture
	Om hur musiken för Music for Airports kom till, och en tanke som han berskriver, där musiken handlade om
	döden och förlika sig med sin dödsrädsla, inför att stiga in i en flygande tub som kan döda en. Att han
	såg flygplatsen som en vänthall för döden. - [ exempel på hur musik kan vara beskrivande ]
	

	"In late 1977 I was waiting for a plane in Cologne airport. (...) I started to wonder what kind of music
	would sound good in a building like that. I thought, 'It has to be interuptable (...), it has to work
	outside the frequencies at which people speak, and at different speeds from speech patterns (...), and it
	has to be able to accommodate all the noises that airports produce. And, most importantly for me, it has
	to have something to do with where you are and what you are there for -- flying, floating and, secretly,
	flirting with death' I though. 'I want to make a kind of music that prepares you for dying'."
	\footnote{Brian Eno, Ambient Music, I \emph{Audio Culture}, Christopher Cox and Daniel Warner (red.),
	London: Bloomsbury, 2017}



Delalande - Modes of listening ( ? ) - om olika lyssnanden, figurativt t ex

Donna Haraway - A Cyborg Manifesto
	Om sammansmältningen mellan människa och maskin, hur vi utvidgar våra egenskaper genom teknik utanför våra
	kroppar. 

Litteratur att kolla in:

Maria Horn - kandidatarbete

Linus Hillborg kandidatarbete

Henrik Frisk om interaktivitet

Joanna Demers - Listening through the noise:
	s. 34: om hur det skedde en splittring i vad musik skulle vara i orkesterkonsertvärlden. Ena lägret
	hävdade att musiken bara fick referera till sig själv, medan det andra såg den som ett narrativt verktyg.
	Hon skriver även att fram till 1980-talet har den här idén dominerat inom utbildningar.  
	

	"The tug-of-war between these two camps imprinted itself on the discipline of musicology, such that until
	the 1980s, most scholarship emphasized the formal or intrinsic aspects of musical works while
	deemphasizing music’s extrinsic and referential qualities."\footnote{Joanna Demers, \emph{Listening
	through the noise}, Oxford : Oxford Univ. Press, 2010}

Mark Fell - One Dimensional Music wuthout Context or meaning:
	En kritik mot Edmund Husserl, som inspirerade Pierre Schaeffers text \emph{Traité des objets
	musicaux}\footnote{Pierre Schaeffer, \emph{Traité des objets musicaux}, Paris: Seuil, 1966}


Jean Baudrilliard (ju högre upplösning av mediet, desto längre upplösning av budskapet)

Jean Molino ( ? )

Mark Fisher - Capitalism - Is there no alternative?

Murray Schafer



John Cage + Rashaan Roland Kirk

La Meme Young - Jessica Ekomane as Guest


3 musiken, beskrivning om musiken och vad för musik som ska innefattas i arbetet

 lite supercollider-musik + ev nåt annat

\end{document}
