\documentclass{article}
\usepackage[utf8]{inputenc}
\usepackage[margin=3cm]{geometry}
\usepackage{fontspec}
\setmonofont[Scale=0.7]{Monaco}
\usepackage{listings}
\usepackage[T1]{fontenc}
\lstset{basicstyle=\ttfamily\footnotesize, tabsize=2} \usepackage[framed, numbered]{sclang-prettifier}

%%%%% Kompilera med xelatex!!!!! %%%%%

%%% TODO : %%%
% [  ] Skriv klart absolut vs programmusik-kapitlet, koppla russolo till kode9 och Emma Frid (diversifiering)
%
% [  ] Musik som Vetenskaplig och konstnärlig forskning - New public management -> formar undervisning och
%	   musiken genom sitt språk. 
%
% [  ] Skriv en frågeställning, förenkla!!! Skriv om det liminala, om transformation och limbo som källa till
%	   inspiration
%
% [  ] Utveckla "figurativiserande"-konceptet och din egen musik, att använda analysmetoden generativt som
%	   kompositionsmetod
%
% [  ] Utveckla vad Moore's Law ära
%
% [  ] Granska Joanna Demers-citatet, vad är det egentligen hon syftar på?
%
% [  ] Använd ett citat ur Linus Hillborgs uppsats som du sedan kan sammanfatta.
%	   Bekräfta hans citat med andra referenser - kika i hans text.
% [  ] Se över Russolo - behöver inte prata om att han var misogyn.
%
% [  ] Skriv Liminal Space-stycket --- Finns det ett liminalt utrymme mellan teknologin och det
%	   extramusikaliska?

% [  ] EVENTUELL FORM:
%		-- Teknologin och Extramusikaliska intryck (Diskussionsdel) ---> Musiken (Resultatdel) ---> Reflektionsdel


\begin{document}
% Vad jag vill prata om och varför jag tycker att det är intressant.
% Jag beskriver varför sökt mig till det här ämnet, utifrån min egen känsla av okunskap kring en viss teknik. 
% Hur jag kunde känna mig begränsad när jag inte behärskade något, och hur det kunde göda en prestationsångest 
% som jag sedan vidarebefordrade till andra. 

\title{Examensuppsats}
\author{Viktor Sandström}
\date{Februari 2022}

	% 1. When a distinguished but elderly scientist states that something is possible, they are almost certainly right. When they state that something is impossible, they are very probably wrong.
	% 2. The only way of discovering the limits of the possible is to venture a little way past them into the impossible.

\section{Liminal Space - An Aesthetic}

\emph{3. Any sufficiently advanced technology is indistinguishable from magic\footnote{\emph{Arthur C. Clarke, "Clarke's Three Laws"}}}



\section{Inledning}
Under min utbildning har det ofta känts som att jag jagat ett ideal av vad elektroakustisk konstmusik är. En
känsla av att inte hinna lära sig alla tekniker som är aktuella inom det elektroakustiska fältet, och att
ständigt göra musik som en på ett eller annat sätt är en del av en lärandeprocess, men som är påeldad av
prestationsångest och en känsla av att inte veta vad man gör. En insikt var att mycket, om inte allt jag gjort
under åren varit något likt en "tech-demo", ett ljudande exempel på vad en teknik kan erbjuda, utan att kunna
ha en känslomässig förankring till materialet. 

Ett exempel är tekniken Ambisonics\footnote{Roger K. Furness, \emph{"Ambisonics-An Overview,"} Paper 8-024,
(1990 May), AES}, som är en populär teknik för att arbeta med "rumsligt ljud" (en anglicism av "spatial
audio", då konceptet är svåröversatt, inte att förväxla med "Apple Spatial Audio"), ett koncept som
innefattar, men som inte är begränsat till surround-ljud, flerkanalsljud och simuleringar av rum. Ambisonics
har fått stort genomslag inom utbildningar i elektroakustisk musik, och på skolor runt om Europa har speciella
konsertsalar byggts för att möta de särskilda kraven som ställs av ambisonics. 


Under åren 2020-2021 genomgick jag en lång sjukskrivning. Jag spenderade större delen av den här perioden i
mitt hem, med undantag för sjukhusbesök, ett julfirande och min trettioårsdag. Under den här sjukdomstiden
började jag vilja skriva musik, mest ur en frustration av att plötsligt ha så mycket tid, det som jag tidigare
klagat över att jag inte haft och använt som ett skäl för att jag inte orkade göra musik. 

Det var först när jag efter hand började må bättre som jag hade ork att börja arbeta med den här idén. I och 
med en förbättring av mitt mående och ett överskott av ledig tid hade jag även bestämt mig för att lära mig
kompositionsverktyget SuperCollider\footnote{\emph{SuperCollider}, https://supercollider.github.io/ (Hämtad
06-02-22)}, som jag tidigare haft svårt med. Därav föll det sig naturligt att mycket av musiken jag skrev var
skriven med detta verktyg. 

Min idé var från början att jag
fattade tycke för de märkliga, bortglömda miljöer som finns på sjukhus och vårdmottagningar, som existerade
mitt inuti en plats som ständigt var i rörelse, men som själva såg obefolkade och stillastående ut. Som att
deras funktionalitet gått förlorad. En sån plats som jag starkt reagerat på var de innergårdar, anlagda med
promenadstråk och icke förvuxna buskar och träd och med monobloc\footnote{\emph{Monobloc (chair)} [wiki],
https://en.wikipedia.org/wiki/Monobloc\_(chair) (Hämtad 06-02-22)}-stolar som nu var täckta av alger, eller 
korridorerna med den offentliga konst som var utplacerad genom sjukhuset. Dessa miljöerna fångade mitt
intresse, men de speglade även min egen tillvaro som, likt dessa platser, kändes stillastående och tidlösa.
Tanken att tonsätta miljöerna på sjukhus, blev istället ett projekt om att tonsätta och dokumentera min egen
stillastående vardag.

Den kompositionsmetoden jag använde kändes annorlunda från min erfarenhet från att skriva musik inom
utbildningens kontext. Musiken var inte ett medel för att uppnå målet att bemästra en ny teknin. Istället
fanns en känsla av att kunna få utforska tekniken om ett behov fanns, eller av nyfikenhet, men där ett
estetiskt uttryck fick vara förgrund för processen.

%%%%%%%%%%%%%%%%%%%%%%%%%%%%%%%%%%%%%%%%%%%%%%%%%%%%%%%%%%%%%%%%%%%%%%%%%%%%%%%%

% Delalande-stycket är inte så fängslande, sätt det efter när jag skriver privat
I Delalands uppsats om lyssnandebeteenden beskriver han tre unika former av lyssnande\footnote{François
Delalande (1998) \emph{Music analysis and reception behaviours: Sommeil by Pierre Henry}, Journal of New Music
Research, 27:1-2, 13-66, DOI: 10.1080/09298219808570738}. Han beskriver dem som det "taxonomiska", det
"empatiska" och det "figurativiserande". Det första, det taxonomiska, är det som enligt Delalande förhåller sig
närmast det traditionellt musikteoretiska analyserandet, att klassificera musikens beståndsdelar, deras
"morfologiska enheters"\footnote{PIERRE SCHAEFFER!!!}. Det taxonomiska intresserar sig främst om relationerna
mellan olika identifierbara ljud och klanger och hur de utvecklas över tid. I kontrast till det taxonomiska,
det empatiska och figurativiserande skiljer sig just i sin tidsuppfattning av musiken och de tolkar inbördes
relationer av ljud. Det empatiska lyssnandet, till skillnad från taxonomiskt, intresserar sig med att
uppfatta musiken horisontellt, det omedelbara samspelet mellan klanger och hur det får en att känna. Istället
för att höra helheten och dess samspel, insatser och uttoningar, försöker man sätta sig i det emotionella
tillståndet som musiken, ögonblick för ögonblick försätter en i. 

% Börja här? beskriv det sista lyssnarbeteendet kort, sedan direkt in på min egen berättelse, för att sedan
% återknyta ordentligt med Delalande.

Det tredje och sista lyssnandebeteendet som Delalande identifierat är det figurativiserande. Den som lyssnar med
detta perspektiv är intresserad av att associera fritt kring de ljud den hör, likt miljöer som beskriver ett
narrativ. Där musiken översätts i lyssnarens huvud till att besrkiva en scen eller plats som lyssnaren
förnimmer i lyssnandet av musiken.

Detta klingade väldigt rätt hos mig när jag börjat arbeta med ett projekt som till viss del handlade om en
sjukskrivning som jag genomgick under vintern och våren av 2021. Det började med att jag fattade tycke för de
märkliga bortglömda miljöer som finns på sjukhus och mottagningar, där det känns som man kan ana en
funktionalitet men som gått förlorad. Platser som någon gång i tiden haft ett ändamål att låta både patienter
och personal tänka på något annat. Den offentliga i korridorer, eller som delade utrymme med
en innergård, anlagd med buskar och träd och med monobloc\footnote{\emph{Monobloc
(chair)} [wiki], https://en.wikipedia.org/wiki/Monobloc\_(chair) (Hämtad 06-02-22)}-stolar som nu var täckta av
alger. Dessa platser präglade denna tid, men det flöt även in i min egen tillvaro, då jag i min sjukskrivning
på något sätt stod still, likt dessa platser. Mitt projekt, som från början var en idé om att tonsätta de här
platserna, blev istället ett projekt om att tonsätta min egen stillastående vardag. 

I den här processen, framförallt när jag mådde bättre, började jag arbeta med musiken, framförallt utifrån
verktyget SuperCollider\footnote{\emph{SuperCollider}, https://supercollider.github.io/ (Hämtad 06-02-22)}.
När jag skrev musiken reflekterade jag kring hur jag skrivit musik tidigare, främst under min utbildning, och
att detta utgångsläge, att gestalta något utifrån ett minne eller känsla, var annorlunda. Under min utbildning
kände jag att jag hade jagat ett ideal av vad elektroakustisk konstmusik är. En känsla av att inte hinna lära
sig alla tekniker är aktuella inom det elektroakustiska fältet, och att ständigt göra musik som en på ett
eller annat sätt är en del av en lärandeprocess, men som är påeldad av prestationsångest och en känsla av att
inte veta vad man gör. En insikt var att mycket, om inte allt jag gjort under åren hade varit något likt en
"tech-demo", ett ljudande exempel på vad denna teknik kan erbjuda, utan att kunna ha en känslomässig
förankring till materialet. 
% /////////

% EN TES: Hur fokuset på absolut musik inom konstmusik, och i instutitioner runt omkring kan ha format
% elektroakustisk musik och dess inställning till teknik. Att försöka hitta något som är universellt och sant, som överskrider subjektet. Exempel på detta är
% sonifieringsgenren, som i sin strävan av att betraktas som en legitim avbildning av data, bortser från de
% konstnärliga val som görs för att få data att passa in i ett ljudande medium. Finns det en koppling mellan den
% här tanken på Absolut musik, musik som existerar enbart som musik (i kontrast till programmusik, musik som
% berättar en extern historia.) och [ [ TEKNIKFIXERINGEN ] ] hos elektroakustiska kompositörer. 

% Har tekniken fått berätta historier åt oss, hur låter de? Hur mycket förlitar sig elektronisk musik på
% möjligheten att gestalta något, och hur mycket lutar den sig på möjligheterna inom teknologier?

% Teknologier har möjliggjort att vi kan skapa extremt avancerade klanger, som är ovanliga eller omöjliga i
% akustiska rum och resonanser, men hur förvaltar vi det och vart ligger vårt ansvar att göra det mer
% tillgängligt?

% En rimligare fråga är kanske hur elektroakustisk musik blivit trots instutitionernas ingrodda åsikt om absolut
% och programmusik?

% Hur har språket om teknik, hur man sett på teknik historiskt och inställningen till den, format hur
% elektronisk musik formas


% koppla till Joanna Demers text. Hur absolut musik förhärskat inom akademien. Kanske koppla till hur många
% utbildningar som jobbar med ambisonics, hur man bygger "ambisonics-kyrkor" runt om i den akademiska
% elektroakusiska världen och vad det gör för musiken. 



% ////////


% Vrida lite på första stycket, börja personligt, sedan skapa en ram i inledningens andra del, t ex Delalande,
% Schaeffer,  Smalley - morfologiska begrepp och teori kring kategorisering av elektroakustisk musik
% Mark Fisher - kritik mot new public management.


\section{State of the art}
  % Som state of the nation - var befinner sig det elektroakustiska fältet just nu? Citat och ev om min egen
  % musik. 

Elektronisk musik har alltid intresserat sig för, och haft ett intimt förhållande till teknologi. Utan nya
upptäckter under 1800-talet och vidare under 1900-talet, hade elektronisk musik som vi hör den idag, inte
varit möjlig. Med detta menar jag uppfinningar som radion, trådlös och trådburen elektrisk signal,
magnetband, mikrofoner, datorer och dess språk samt digital signalprocessering\footnote{DSP, Digital Signal
Processing} etc. Studion som instrument har alltid varit central inom elektronisk musik, idag mer än
någonsin, med priset av en persondator som ständigt sjunker som en produkt av en kapplöpning mot "Moore's
law"\footnote{Moore's Law, Ref....}, processeringsförmåga och nya billigare tillverkningsmetoder. Idag kan
en produktionsstudio för hemmabruk rymmas i en dator som ryms på ett enda kretskort, som Raspberry
Pi\footnote{Raspberry Pi [https], \emph{https://www.raspberrypi.com/}}, med ett pris som ligger mellan \$15
- 35. Tillgängligheten till en studio, som tidigare var begränsad till instutitioner som nationella
radiostationer har gradvis hamnat inom räckhåll för individens händer.

  % Demokratiseringen av studion.
I och med att denna teknik förflyttats närmare individen har den demokratiserats. Musikteknik har i teorin
aldrig varit så tillgängligt. Dessa hjälpmedel för att skriva, spela in och producera kan ibland rymmas i
fickan på sin användare.

Trots denna tillgänglighet finns det fortfarande barriärer. Hinder i form av svårtillgänglig information,
eller avskräckande konfigurationssteg som förutsätter en hög grad av förkunskap för att du ska kunna få
tillgång till funktionaliteten du egentligen vill komma åt. Med detta syftar jag på den tidigare nämnda
Raspberry Pi-mikrodatorn, som trots sitt överkomliga pris förutsätter att du är bekväm att lämna etablerade
operativsystemsparadigmer som Windows och MacOS, kunna utföra kommandon i en "shell"-miljö, och kunna
felsöka när installationer eller konfigurationer inte genomförs som de ska i tidigare nämnda shell-miljö,
eftersom dessa kommandon kan vara mycket obskyra för den oinvigde. 
	
  % Skyll på att det är den fria marknadens fel att vi inte använder open source
På detta vis är vi också fast i en marknad där dessa open source-verktyg drunknar på grund av sin egen
otillgänglighet, verktyg som annars skulle kunna öppna dörrar för många till elektronisk musik. Denna
marknad består av musikteknologi som säljs för höga summor, något som med open source-teknologi skulle kunna
vara mycket lätt att ersätta, men som marknadsförs till oss som en "easy fix", något som inte bara
underlättar och överbryggar tröskeln till det tekniska, men också till det estetiska. Dessa verktyg
säljs med en idé att om du bara skaffar dig denna sista sak, denna pryl, så kommer kreativiteten flöda.
Tidigare banbrytade verktygstillverkare, som Moog, eller Dave Smith Instruments, säljer nu sina instrument
om hi-fi-utrustning. De är brukbara och låter väldigt bra, men de är så dyra att priset stänger dörren för
den som som försöker ge sig in i musiken, eller är demoraliserande för den som tänker att verktyget kommer
lösa dess problem. De är otillgängliga ekonomiskt men ur ett popkulturellt perspektiv mycket tillgängliga,
eftersom populära verktyg blir synonyma med ett ljud, en stil, en kultur och med musikalisk och ekonomisk
framgång. Ett exempel på detta är den legendariska TB303\footnote{TB303}-synten från Roland, som genom sitt
stilbildande ljud som idag är synonymt med elektronisk dansmusik, som idag är värld säljs för mellan 2 - 5
gånger sitt säljpris, år 1982 (Originalpris: \$395.00 . Efter inflationskompensation: \$1,161.32. På Ebay
2022: \$5,543.64).

  % otillgängligt för majoriteten av musikutövare, och kan istället existera som en statussymbol mer än ett
  % kreativt verktyg. 

  % vänd på det och säg att inom EAM så känns det ofta som man har en motsatt inställning, 
  % är motståndet att vi förkastar konventionella teknologier för svårtillgänglig, och har det då varit
  % skadligt för EAM i stort?

Inom elektroakustisk musik kan snarare en motsatt reaktion uppstå. Här ges de mer obskyra verktygen spelrum,
och blir i sig en statussymbol i sin otillgänglighet. De tidigare nämnda verktygen som är svåra för den
vanliga brukaren av musikteknologi blir här ett verktyg för att visa sin virtuositet. 
är datorprogrammering. 
Istället för att använda färdiga verktyg för att åstadkomma ljud, och applicera sin kunskap om olika
syntesmetoder på en mjuk- eller hårdvarusynt, kan du nu skriva ljudprocesseringsalgoritmer själv och på egen
hand hushålla med din dators resurser. Visserligen möjliggör sådan teknik större flexibilitet och en
inspirationkälla i att utforska tidigare onåbara delar av ett ljudprogram, icke-linjära strukturer av musik
och möjligheten att påverka ljud i ett "close to the metal"-perspektiv, bearbetningar av dess data och form
på en COMPUTATIONAL nivå.
Joanna Demers skriver i "Listening Through The Noise":


\begin{quote}
The sheer freedom of electroacoustic music constitutes both its strength and its burden. With the
latitude to use both conventional musical figures and random, seemingly unintentional sounds,
electroacoustic composers have generated an amount of theoretical literature concerning the act of
listening that is unrivaled in any other genre of music.\footnote{Listening Through The Noise}
\end{quote}


Joanna Demers pratar om den oslagbara friheten hos elektronisk musik. Klangvärlden hos elektronisk musik har
till synes oändlig potential. Detta kan dock vara till dess nackdel. I Linus Hillborgs uppsats
PLACEHOLDER\footnote{linus hillborgs uppsats}, skriver han om teknologiers begränsingar, hur det kreativa
kan komma ur möjligheternas begränsning. Genom att utforska var den gränsen går kan man även hitta de
särskiljande kvalitéer som möjliggör något som är unikt för den teknologin. Det kreativa arbetet kan då likas
vid att känna sig fram i blindo längs en ojämn yta för att lokalisera och kartlägga det intressanta, samt att
lyfta fram det i ljuset. 

% Koppla Robin James idée om resilience till fortsatt marknadsföring av instrument som nyckeln till
% musikskrivning. Att vi internaliserar det och använder oss av den marknadsföringen mot varann. 

% Robin James presenterar i sin bok Resilience and Melancholia\footnote{\emph{Robin James}, Resilience and
% Melancholia, Zero Books, !!!!!!!} en slagkraftig analys på biopolitik och Neo-liberalism genom en lins av 
% elektronisk musik. 

% Open Source? 

% DIY

\section{Diskussionsdel}
\subsection{Absolute vs Programmatic music.}
% - Joanna Demers - Listening through the noise: Om fokuset på absolut musik i högre
%   musikutbildningar, och hur programmusik har hamnat i skymundan.


I sin text Listening Through the Noise diskuterar Joanna Demers hur man som elektroakustisk kompositör kan
förhålla sig till extramusikaliska referenser. Hon hävdar att inom konservatorier har historiskt sett den 
"absolut" musik fått stort spelrum. Den har ansetts som norm, och har varit det som undervisats på
utbildningar i komposition. Det är en modernistisk stil som speglar den modernistiska genikulten, där musiken
eller konsten i stort uppstår ur geniets sinne, gärna frånkopplat från yttre störningsmoment. Likaså är den
"absoluta" musiken en musik vars mening och syfte är sprunget ur dess egen genialitet, istället för att
hänvisa eller berätta om något extramusikaliskt. 

Med extramusikaliskt menas något som ligger utanför musiken, "lying outside the province of
music"\footnote{Merriam-Webster, \emph{https://www.merriam-webster.com/dictionary/extramusical} (Hämtad
29-03-2022)}, strikt tolkat allt som inte innefattas inom det fysikaliska i ljudvågor som sammanfaller i
luften, eller som innefattas inom de regler för ljud-, form- och harmonilära används för att beskriva musik. 


\begin{quote}
		Electroacoustic musicians are also interested in whether music can, does, and should refer beyond
		itself to the outside world, which is not surprising, since so many electroacoustic musicians are
		trained within the Western artmusic tradition. In fact, the theoretical positions that many
		electroacoustic musicians espouse recall the longstanding formalist/hermeneutic debate in both
		nonelectronic Western art music and musicology. But the translation from traditional to plugged-in
		instruments is not without its snags. Unlike nonelectronic music, electronic music can transcend
		timbral limitations and just about every other existential limitation of traditional musical
		discourse. Electronic music can incorporate sounds of the outside world with ease and can generate new
		timbres that defy identifi cation as music (or anything else). As such, electronic music can be
		mimetic and representational or abstract and obscure. In this new terrain, old theories about music’s
		separation from (or dependence on) culture and history may no longer be pertinent. Instead, we need to
		talk about sounds themselves, not only as units of musical syntax but also as sounds. Are sounds
		always referential? Is it possible to hear sound before it has been laden with the associations of
		culture, history, or society? In short, does the act of listening to electronic music
		necessarily involve relating sounds to what is already familiar? If not, how do we make sense of the
		experience of listening to abstract, nonreferential, unrecognizable sounds?\footnote{Listening Through The Noise}
\end{quote}

Hon frågar här ifall de ljud som frammanas inom elektroakustisk och akusmatisk musik kan existera utanför sig
själva, vara självrefererande, eller om de kan existera, likt Pierre Schaeffers teori om
ljudobjekt\footnote{PIERRE}, som något ursprunget inte längre är urskiljbart. Schaeffer myntar uttrycket
reducerat lyssnande för att beskriva detta fenomen, som gör att lyssnaren blir fri från ljudets inbyggda
associativa kraft och nu existerar i sin mest avskalade, "rena" form. Hon ställer hans teori kring ljudobjekt
emot Trevor Wisharts omtolkning av begreppet. Demers skriver: "\emph{Wishart fells that it is impossible to separate
sounds from their associations, so it is incumbent upon the composer to acknowledge and work with sound
references rather than repress them}". Wishart är i On Sonic Art\footnote{Trevor Wishart (1996), \emph{On Sonic
Art}, Amsterdam: Harwood} skeptisk till Schaeffers reducerade lyssnade. Wishart menar att ljud aldrig kan vara
frånskiljda sin associativa kraft, och lägger istället bedömningen om ett ljuds associativa, extramusikaliska
egenskaper hos lyssnaren, mottagaren av det musikaliska meddelandet. 

Wishart delar istället begreppet i reella och icke-reella ljud, samt reella och
icke-reella rum.


och Trevor Wishart som omformulerade kring
ljudobjekten, som alltid innehåller en extramusikalisk struktur. OSV OSV ---> FYLL UT!!! koppla till det
extramusikaliska och till LIMINALITET

% Russolo - fokus och intresset kring krigsljud och krigsmaskiner, något som ligger till grund för tidig eam,
% som är manligt kodat och fascistiskt.


I futuristiska manifestet\footnote{futuristiska manifestet}, skriver Filippo Tommaso Marinetti om den nya
konsten som ska upphöja och hylla det nya, motorer och hastigheten i industri och urbanisering, samt till
kriget, som är "världens enda hygien". Det är aggresivt misogynt och antifeministiskt, vilket stämmer in på
dess koppling till fascismen och dess dyrkan av styrka och militarism, samt dåtidens kvinnosyn.
Luigi Russolo, som var en kompositör inom den futuristiska rörelsen, ansåg att det var rättighet och en
skyldighet att skriva musik som okunnig. Han skrev i Oljudets konst\footnote{Oljudets konst, Lárte dei
rumori} att det var först då vi kunde komma undan akademien och etablissemangets hårda grepp runt vilken
musik som fick tillverkas. Att även man skulle sluta imitera gamla tonsättare och tekniker och göra dagens
musik.


% Detta är den av de få
% punkter från futuristerna manifest som kan ha någon vikt idag. Dock lyder den fullständiga formuleringen
% (punkt 10 i manifestet):


% "Vi vill förstöra museerna, biblioteken, akademier, av alla slag och bekämpa moralism, feminism och varje
% opportunistisk eller utilitaristisk feghet."


Luigi Russolo lät bygga ett antal musikaliska maskiner, som skulle reproducera stadens ljud i en
konsertsättning. Dessa kan ses som ett tidigt exempel på elektroakustisk musik, men också på musik som ville
bryta mot den, som Joanna Demers beskriver, förhärskande "absoluta" musiken som lärdes ut på
konservatorier. Futuristerna ville på ett väldigt bokstavligt sätt få sin musik att handla om något utanför
sig själv. 


% Med detta exempel vill jag prata om hur elektroakustisk musik i sitt ursprung ofta tagit ett uttryck i
% programmusik, och även ett avständstagande från den absoluta musiken. 

% Russolo och futuristerna på början av 1900talet var intersserade av maskiner, industri och krig, med tydliga
% fascistiska konnotationer. Deras arv kan fortfarande höras inom den elektroakusiska genren, med sitt
% intresse för ljud som kommit som biprodukt ur industri och från maskiner, och stundtals rent aggressiva
% ljud, som t ex noise. 


% Emma Frid - Diverse sounds, kap 2.2 music diversity and inclusion - "Varför ska vi sträva efter diversity?"
% s. 18, empowerment through music and technology

% Kode9 och hans avhandling där han myntar begreppet "sonic warfare", en återkoppling till fascinationen
% kring vapen, men här omvänt ur en granskande kritiserande blick från en elektronisk kompositör.


\subsection{Musik som vetenskaplig forskning}
- vad har det haft för konsekvenser för hur och vilken musik som skrivs, när materialet på utbildningar filtreras
  genom ett behov av att motivera sig som ett akademiskt ämne. Att detta fokus på att få vara vetenskaplig
  forskning har satt sitt spår i vad och hur man lär ut komposition, och främst elektronisk musik som är så
  fokuserad på teknik. 

  Mark Fishers tankar om hur utbildningar behöver motivera sin egen existens, kanske citatet om vilket
  förhållande lärare (instutition) och elev har, vem som köper en tjänst av vem. Att högre utbildningar
  köper en tjänst, behöver ha elever som läser för att få ekonomiskt stöd, istället för att eleven köper sin
  utbildning av instutitionen. 

  Robin James - Resilience and Melancholy

	
  % Neo-liberalism och New Public Management

  % Edu-art - konst som är producerad inom akademin, som inte har något värde utanför den världen.

  % som konstnärer behöver man inte ge ifrån sig något som är bra.
  % som forskare så kan man inte tweaka 

  % Kölnstudion
  % Parisstudion

  % I came into the studio to make noises speak, I stumbled into music - Pierre Schaeffer

  % Henk Borg - Frailing

  % on, for or in the arts: 
	% on: musikvetenskap, musikhistoria - forska på musiken själv
	% for: barockinterpretation - syftet är att spela barock
	% in: 

	% konstnärlig forskning vad är det?

	% Henri Bergson - Introduction to metaphysics

  % Denis Smalley - om division of labor

  % Simon Emmerson - The relation of language to materials

  % Ens subjekt, ens röst, som man odlar, kontra t ex AI-musik, där du kodar bort ditt eget subjekt
  % I want to hear someones music, not someones max-patch. i AI är det kanske omvänt
  % AI är kanske den utlimata kritiken mot det jag kritiserar. 

  % Åsa Stjerna - Before Sound - om att släppa kontrollen om musiken 
  % Cybernetic manifesto

\subsection{Teknik som inspiration}
- Att inspireras av teknik, dess möjligheter och dess begränsningar. Detta är en viktig rubrik, då det är ett
  en viktig del av elektronisk musik, och en viktig del av mångas metod. 

  Linus Hillborgs examensarbete. exemplet med metal gear solid

\subsection{Tekniken som en börda}
- Arbetssätt och teknologi som en merit, där ju otillgängligare något är, eller mer obskyrt, desto större vikt
  har det inom elektronisk musikkretsar. 

  Konceptet gate-keeping, att hålla på "the fruits of your labor" istället för att dela kunskap, för att ha
  ett övertag över ens jämlikar.

  Maria Horns examensarbete - Om manligt och kvinnligt kodade egenskaper, och hur tekniken varit något som män
  länge fått höra att de har rätt att använda, för de har rätt att kontrollera världen, medan kvinnor inte har
  det. Detta har bidragit till att kvinnor inte har lika lätt in i rum som dominerats av män, men en
  tekno-jargong som passivt (eller aktivt) exkluderar kvinnor. 

  Fokuset på att bygga sitt verktyg, och att man kan gå vilse i den processen. Jean Baudrillard, högre
  upplösning av mediet, lägre upplösning av innehållet(?). 
  La Meme Young-podcasten med Jessica Ekomane, där de pratar om den hypotetiska masterstudenten som spenderar
  2 år att bygga sin enorma \emph{patch}, som ska lösa alla problem och äntligen möjliggöra den där musiken
  som hen alltid drömt om, för att sedan gå vilse i syftet och tappa siktet på slutmålet, att göra musik. 

  % t ex stycken som brukar extended techniques, men utan att veta varför
  % Buchla - "we are not limited by our tools but our mindsets"

  % vad är antitesen till "megapatchen"? att aldrig spara något? att göra något i stunden (live-kodning)? 
  % en skillnad mellan oss och andra instrumentalister.


\section{Koppling till musiken}
- Koppla till min musik, till EPn som jag gjorde som start för examensprojektet, och även de experiment som
  jag gjorde med ljud på sample-nivå. 
- Började med en idé om att tonsätta och dokumentera platser, de jag såg under min sjukskrivning, framförallt
  de övergivna offentliga rum som fanns i de miljöerna. Hur det projektet övergick till att dokumenter min
  egen vardag, som på något sätt länkades ihop med de platserna. Beskriv hur begreppet
  liminalitet\footnote{liminalitet} kan användas för att beskriva båda dessa, då det beskriver att vara i
  tröskeln av någonting. En tröskelplats, en plats på vägen mellan två platser, eller en plats i en
  övergångsperiod, på väg till att bli något annat. Eller ett tröskeltillstånd, som i ritualen, där det finns
  ett före och ett efter, men under ritualen befinner man sig i ett mellanläge. Hur min situation som
  sjukskriven och bunden till mitt hem kunde ses som ett sådant tröskeltillstånd.

  Den ljudande delen av mitt examensarbete består av en EP släppt på kassett på bolaget
  Kalkatraz\footnote{länk till kalkatraz nånting}, samt en ljudinstallation, som i en omarbetning även blev
  ett stycke för piano och elektronik.

  Min EP skrevs i huvudsak under min tid som sjukskriven efter min njurtransplantation. De fyra spåren har
  namn som refererar till tidpunkter eller platser som fastnat tydligt i minnet hos mig. 
	\subsection{DEC}
	"Dec" var det första stycket jag började arbeta med. Ursprungligen var detta ett stycke som skulle vara en
	form av ljudläggning av platser på sjukhus ( se inledningen ) som känns bortglömda, med fokus på
	innergårdar och utrymmen runtomkring och mellan husen, ofta med ett gråaktigt, deprimerande eller illa
	skött offentlig konst. En komposition om att spendera mycket tid i dessa miljöer. Dock övergick projektet
	efter hand till att beskriva den första tiden då jag var sjukskriven, när jag abrupt pausade mina studier
	och påbörjade en tillvaro som var ett typ av mellantillstånd, där varje dag smälte ihop i nästa, avbrutna
	av läkarbesök. 
		Jag började arbetet med att fundera över signal- och styrflöde i ett modulärt system jag nyligen hade
		byggt. Jag hade en plan om att kunna styra synten med MIDI\footnote{MIDI} från SuperCollider, och att
		den sedan skulle efterarbetas i Reaper:
		\begin{center}
			% tabular behöver ytterligare argument som visar hur många celler 'c'
			\begin{tabular}{ c c c c c }
				% avgränsningen görs med '&' mellan celler och '\\' mellan rader
				Midi		  & -> & Ljud			& -> & Processering/mastring \\
				SuperCollider & -> & Eurorack-synth & -> & Reaper
			\end{tabular}
		\end{center}

		Till en början var arbetet att få en lösning med MIDI för att skicka ett playback av tonhöjd- och
		durationsmaterial till synten. Detta var något nytt för mig och jag fick det inte att fungera som jag
		ville (inte alls), så jag övergav idén för intern playback i synten själv. 
		Jag arbetade med ett QPAS-filter (Quad Peak Animation System) från tillverkaren Make
		Noise\footnote{makenoise}. Med detta filter kunde jag arbeta med ytterligare en stämma, genom att
		framhäva och finstämma vissa övertoner i den styckets klang. Stämman bidrog till variation i en
		annars väldigt lunkande och stillastående ljudbild, Jag upplevde att den bidrog till mer motrörelse,
		trots att stämman var en en ton med en statisk relation till de underliggande tonerna. Med denna
		parameter samt möjligheten att bredda och öppna filtret för att släppa igenom mer ljud, och varsin
		wavefold-effekt på oscillatorerna, hade jag möjlighet att framföra stycket med ett mått
		"liveness"\footnote{citat om liveness i elektronisk musik}. Ytterligare hade jag en bruskälla och ett
		LFO\footnote{LFO} som gick ut från från SuperCollider till synten, för att tillföra ett mått av slump.
		Jag kunde styra den från samma kontroll som de andra. Detta kändes viktigt då jag tanken var att
		spela musiken som en solokonsert, och med tanken på min bakgrund som pianist kändes det rätt att ha
		något att påverka musiken med. 

	\subsection{Koltrast}
	Stycket "Koltrast" skrevs under sommaren 2021, när jag spenderade tid i Göteborg. Stommen i stycket är en
	lång field-recording av koltrastsång. Efter min operation hade jag haft problem med att sova, och
	inspelningen gjordes under en natt som var så varm att det var tvunget att ha balkongdörren öppen. Precis
	när det började ljusna vaknade jag av att det lät som fåglar inuti rummet. Det visade sig vara en handfull
	koltrastar som sjöng över hustaken på Doktor Fries Torg. Fångade det genom att plocka fram en
	Zoom\footnote{portastudio Zoom} innan jag gick och lade mig igen. 
		När jag sedan bestämde mig för att skriva ett stycke kring fågelsången, ville jag testa om jag kunde
	låta inspelningen styra musikaliska parametrar på något vis. Min första tanke var att använda
	FFT-analys\footnote{FFT-analys} för att kunna följa fågelsångens egen tonhöjd, och tillåta den styra
	tonhöjden av en synt. Under processen hade jag blev jag dock mycket fäst vid hur inspelningen lät i sig
	själv och ville inte bygga ett instrument som skulle göra en konstgjord kopia, om än möjligtvis
	intressant. Istället bestämde jag mig för att bygga ett program i SuperCollider som kunde följa amplituden
	av fågelsången, och ta beslut kring musikaliska händelser när amplituden översteg ett tröskelvärde. Synten
	som triggades av dessa event hade en tydlig attack, men om dess eget volymenvelop öppnades upp och tillät
	för längre toner påminde det om inspelningar jag hört av orgelpipor, där mikrofonen stått mycket nära
	mekanik och klaffar. Jag bestämde mig då för att det skulle bli något i stil med ett syntetiskt
	orgelstycke. 

	%%%% SÄTT IN LITE SC-KOD HÄR %%%%

	Jag undersökte även "physical modelling", tekniken att framställa ljud som liknar akustiska klanger på
	syntetisk väg. Till detta använde jag en färdig klass i SuperCollider som heter DWGBowedTor från
	biblioteket DWG\footnote{DWG}. Tillsammans med en basstämma fick agera kontrapunktiskt mot den mer
	stokastiska koltraststämman och dess ackompanjerande synt, de fick vara mer förutsägbara och ha en mer
	cirkulär form. Här arbetade jag även med att skriva sekvenser med "kontrollstrukturer"\footnote{SC
	controll structures}, och göra enkla sekvenser som varierade sig baserat på vilket varv vi var på i
	loopen. 
	\pagebreak

\begin{lstlisting}[style=SuperCollider-IDE, caption=Amplitudtriggad synt]
SynthDef(\demandSynth2, {
	var fund = 69;
	var trigger, in, sig, env, fade, verb, fadeOut;

	in = A2K.kr(In.ar(\in.kr(55), 2) * 4);
	fade = Line.kr( 0, 1, \fadeTime.kr(2));
	trigger = Trig.kr(			    // <-- Här sker triggningen
		InRange.kr( 
			Amplitude.kr(
				A2K.kr(in.lag(1) * 4), 0.2, 0.5
			), \loTresh.kr(0.01), 0.8 
		).lag(0.01)
	);

	env = EnvGen.kr(
		Env( 
			[ 0, 0.01, 1, 0 ],
			[
				0.05,
				\atk.kr(0.1).linexp(0, 1, 0.01, 0.3),
				\rel.kr(1).linexp(0, 1, 0.01, 0.39) 
			], curve: -4
		), trigger
	);

	sig = DPW3Tri.ar( Demand.kr(trigger, 0, demandUGens: Dseq( [ 
			[ fund, (fund*3) / 2], // Db, Ab
			[ (fund*5) / 4, (fund*5) / 3], // F, Bb
			[ (fund*15) / 8, (fund*9) / 8], // C, Eb
			[ (fund*3) / 2, (fund*45) / 32],  // Ab, G
			[ (fund*15) / 16, (fund*5) / 4]
			] * 2, inf
		) * Dwrand([1, 2, 0.75], [0.6,0.2, 0.1], inf)).lag(0.1)
	);

	sig = sig * fade * EnvFollow.kr( in ).lag(0.7) * \vol.kr(0.7) ;
	sig = HPF.ar(sig, \hpf.kr(440));
	fadeOut = Line.kr(1,0,\fadeTime.kr);
	verb = NHHall.ar(
		sig, 
		\verbTime.kr(4).linlin(
			0, 1, 0, 12
			), 
		0.5, 800, 0.5, 2000, 0.2, 0.2, 0.3
	);

	Out.ar( \out.kr(0), 0.8 * env * sig!2.tanh + ( 
		verb * \verbVol.kr(0.3).linexp(0, 1, 0.01, 0.6))
	);  
}).add;
\end{lstlisting}


  I exemplet ovan visar jag min lösning på hur jag triggade synten på amplitud (på rad 7), samt nedan hur jag
  hur jag använde många styrparametrar i varje ljud för att kunna påverka ljuden i realtid. Objektet "f"
  representerar min midikontroller, och "f.faderAt(1)" motsvarar fadern på plats 1.


\begin{lstlisting}[style=SuperCollider-IDE, caption=Syntens kontrollschema]
~follow2 = Synth(\demandSynth2, 
	[ \loThresh, f.faderAt(1).asMap, // asMap, annars används bus-numret som värde,
		\atk, f.faderAt(2).asMap,		   // kan bli apstarkt!
		\rel, f.faderAt(3).asMap,
		\verbTime, f.faderAt(4).asMap,
		\verbVol, f.faderAt(5).asMap,
		\vol, f.faderAt(6).asMap,
		\hpf, 350],
		addAction: \addToTail); 
\end{lstlisting}

  \subsection{K87-89}
  K87-89 heter sjukhusavdelningen på Huddinge sjukhus där jag och min pappa var efter min operation. Den här
  perioden, och de kommande veckorna präglades av en väldigt hög energinivå hos mig själv, delvis på grund av
  ett mycket förbättrat fysiskt mående, men också definitivt på grund av de starka mediciner jag fick.
  Kopplingen till det liminala var att vi, mitt under Covid19-pandemi, var tvungna att stanna på sjukhus en
  vecka, med bara varandra och sjukhuspersonal som sällskap. Detta i kombination med hur en operation är en
  slags ritual, en exorcism av det onda. 

  Jag ville fånga denna post-ritualistiska extas av febrig energi i ljud, vilket slutligen resulterade i detta
  stycke. Eftersom jag har en förkärlek för långsamma klanger var min första tanke inte att göra något
  rytmiskt intensivt. Jag började istället att försöka skapa ett ljud som kunde gestalta ett sprak av
  elektrisitet, som konstant överladdning av synapserna, något böljande. Ett ljud som var ostadigt och
  lynnigt. 

  Ljudet jag kom fram till hittade jag delvis av en slump. Jag experimenterade med ett objekt i
  SuperCollider vid namn "Fold", som tog vågformer över en viss amplitud och vände dem inåt mot sig själva
  180° ( se bild ). Det tar en signal som input, samt två värden för tröskelvärdet på den positiva samt
  negativa polen av signalen. Ljudet blev till när jag av misstag satte en av parametrarna till '0', vilket
  vände allt ljud in på sig själv i oändlighet, och ena polen av ljudet blev i praktiken bortklippt. Det
  intressanta med detta var att under vissa omständigheter kunde ljudet få en DC offset\footnote{DC offset}
  och lägga sig i den bortklippta polen, vilket resulterade i fullständigt bortfall av klang, för att sedan
  bryta igenom som från bakom en vägg. Jag rättade till mitt miss men effekten fanns kvar, dock inte lika
  markant, men framträdde mer när jag när jag hade flera i följd, när ljudet passerade genom effektkedjan.

\begin{lstlisting}[style=SuperCollider-IDE, caption=Fold-synt]
SynthDef(\fold, {  // \t_trig, \freq, \fold.
	var sig, env, out, verb, dirt;

	env = EnvGen.kr(
		envelope: Env(
			[0,1,0], [\atk.kr(0.4), \rel.kr(2.6)], curve: \lin
			), 
		gate: \t_trig.kr(0), levelScale: 1, timeScale: 1, doneAction: 2);

	dirt = LFDNoise3.kr(25);

	sig = LPF.ar( 
		Fold.ar(
			XFade2.ar(
				SinOsc.ar(\freq.kr(64).midicps), 
				DPW3Tri.ar(\freq.kr.midicps), \fold.kr(0.1).linlin(0, 1, -1, 1) + dirt.linlin(0, 1, 0, 0.1)
			),
			-1 * (\fold.kr + dirt.linexp(0, 1, 0.01, 0.05)),
			\fold.kr + dirt.linexp(0, 1, 0.01, 0.05)
		) * env, 5000 );

	Out.ar(\out.kr(0), Splay.ar(sig*\vol.kr(0.1).linlin(0, 1, 0, 0.35))).tanh; // maybe over-did it with .tanh?
	Out.ar(\out.kr+2, Splay.ar(sig!2*\vol.kr).linlin(0, 1, 0, 0.35)).tanh;

}).add;
\end{lstlisting}

  \subsection{Metamorfos}
  Under min sjukskrivning hade jag fått uppmaningen att gå promenader, och jag passade på att spela in ljud i
  min omgivning. Dessa ljud använde jag senare i ljudinstallationen "Metamorfos", som ställdes ut en helg i
  februari 2022 på Galleri Resorb i Stockholm. Ljudinstallationen hade börjat som ett experiment på
  filstrukturen hos en ljudfil, om det skulle vara möjligt att göra bearbetningar på den som skulle vara
  intressanta. 

  Mitt första experiment var att skriva en "disintegration loop"\footnote{William Basinsky} (kända
  exempel är William Basinskys "Disintegration Loops", eller Alvin Luciers "I am sitting in a room"), som
  tömde datastrukturen på innehåll, nollställde varje sample en och en. Effekten blev att jag långsamt
  tillintetgjorde det inspelade ljudet i filen. Transienter hos ljudet försvann och ljudfilen fick en mer och
  mer homogen textur.

  Jag testade sedan samma teknik, men istället för att "tömma" en ljudfil använde jag metoden för att göra en
  övergång mellan två ljudfiler, en gradvis transformation från en fil till en annan. Jag fann dock att
  transformationen var så gradvis att när processen var halvvägs mellan den ena och den andra filen var dess
  ljud svårt att åtskilja från vitt brus. Min slutgiltiga version tog därför större bitar, "chunks" av
  kontinuerliga samples, där ljudfilerna gjorde hårda klipp in i varann. Jag fann att med ett antal ljudfiler
  som blandades in i varandra kunde en effekt uppstå där det kändes som ljudfilerna spelades parallellt, trots
  att det var omöjligt med tanke på hur programmet var uppbyggt. Vad som hände var en psykoakustisk effekt som
  jag fann vara mycket intressant. Distributionen av dessa "chunks" var slumpmässig och beslut togs om
  distributionen efter varje genomspelning av ljudfilens fulla längd. Eftersom jag arbetade med ett antal
  ljudfiler, bestämde jag mig för att stycket inte skulle ha någon slutdestination. Det var mer intressant att
  rotera ljudfiler efter varje genomspelning, och införa nytt material i det reda upphackade materialet. Det
  bidrog till en ständig rörelse, men också en dekonstruktion av ljudmaterialet, och jag tänkte på något jag
  läst om fjärilar när de förpuppas. Fjärilslarven löses upp av enzymer och blir till en soppa av proteiner,
  som sedan byggstenarna som bygger fjärilen.

\begin{lstlisting}[style=SuperCollider-IDE, caption=Funktion för "metamorfos"]
~metamorphosis = {|larv, fjril, chunk, x = -1|
	var rand = rrand(0, larv.size);
	var i = 0;
	if (x == rand && larv[rand] == fjril[rand]) {
		thisFunction.value(larv, fjril, chunk, rand);
	} {
		for(0, chunk, {|i|
			larv.wrapPut((rand + i), fjril.wrapAt((rand + i)));
		});
	};
};
\end{lstlisting}
  % (Junk Space - tuffa arkitekten som Malte nämnde)

  % Skriv även om hur mitt tröskeltillstånd möjliggjorde att jag orkade lära mig saker som jag tidigare haft en
  % stress för att lära mig, de i mina ögon upphöjda verktygen som jag tidigare hade kastat mig ut för att göra
  % musik med, utan att riktigt veta hur man gjorde, och utan ett lugn kring vad jag ville göra.

  \end{document}
