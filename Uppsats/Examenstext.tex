\documentclass{article}
\usepackage[utf8]{inputenc}
\usepackage[margin=3cm]{geometry}
\usepackage{fontspec}
\setmonofont[Scale=0.7]{Monaco}
\usepackage{listings}
\usepackage[T1]{fontenc}
\usepackage[swedish]{babel}
\renewcommand{\baselinestretch}{1.5}
\lstset{basicstyle=\ttfamily\footnotesize, tabsize=2} \usepackage[framed, numbered]{sclang-prettifier}

%%%%% Kompilera med xelatex!!!!! %%%%%

%%% TODO : %%%
% [  ] Skriv klart absolut vs programmusik-kapitlet, koppla russolo till kode9 och Emma Frid (diversifiering)
%
% [  ] Musik som Vetenskaplig och konstnärlig forskning - New public management -> formar undervisning och
%	   musiken genom sitt språk. 
%
% [  ] Skriv en frågeställning, förenkla!!! Skriv om det liminala, om transformation och limbo som källa till
%	   inspiration
%
% [  ] Utveckla "figurativiserande"-konceptet och din egen musik, att använda analysmetoden generativt som
%	   kompositionsmetod
%
% [  ] Utveckla vad Moore's Law ära
%
% [  ] Granska Joanna Demers-citatet, vad är det egentligen hon syftar på?
%
% [  ] Använd ett citat ur Linus Hillborgs uppsats som du sedan kan sammanfatta.
%	   Bekräfta hans citat med andra referenser - kika i hans text.
% [  ] Se över Russolo - behöver inte prata om att han var misogyn.
%
% [  ] Skriv Liminal Space-stycket --- Finns det ett liminalt utrymme mellan teknologin och det
%	   extramusikaliska?

% [  ] Fundera på Arthur C Clarke citatet, vad har det för koppling till det jag vill säga. 

% [  ] EVENTUELL FORM:
%		-- Teknologin och Extramusikaliska intryck (Diskussionsdel) ---> Musiken (Resultatdel) ---> Reflektionsdel

% Tänk på viktningen mellan extramusikaliska referenser och tekniken när din egen musik beskrivs. 

%	

\title{%
	Liminal Space - An Aesthetic \\
\large{Vad har extramusikaliska referenser för roll i det elektroakustiska komponerandet?}
%
\large{Har extramusikaliska referenser en plats i elektroakustisk musik?}
}
% \author{Viktor Sandström}
% \date{Februari 2022}

\begin{document}
\maketitle
\tableofcontents
\newpage
% Vad jag vill prata om och varför jag tycker att det är intressant.
% Jag beskriver varför sökt mig till det här ämnet, utifrån min egen känsla av okunskap kring en viss teknik. 
% Hur jag kunde känna mig begränsad när jag inte behärskade något, och hur det kunde göda en prestationsångest 
% som jag sedan vidarebefordrade till andra. 


	% 1. When a distinguished but elderly scientist states that something is possible, they are almost certainly right. When they state that something is impossible, they are very probably wrong.
	% 2. The only way of discovering the limits of the possible is to venture a little way past them into the impossible.

\begin{center}
	\hspace{0pt}
	\vfill
\emph{Any sufficiently advanced technology is indistinguishable from magic\footnote{\emph{Arthur C. Clarke, "Clarke's Three Laws"}}}
	\vfill
	\hspace{0pt}
\end{center}
\newpage


% Detta kan ses som programbladet till musiken som innefattas av den konstnärliga delen av examensarbetet.


\section{Inledning}

% ny inledning:
Under åren 2020-2021 genomgick jag en lång sjukskrivning. Jag spenderade större delen av den här perioden i
mitt hem, med undantag för sjukhusbesök, ett julfirande och min trettioårsdag. Under den här sjukdomstiden
började jag vilja skriva musik, mest ur en frustration av att plötsligt ha så mycket tid, något som jag tidigare
klagat över att jag inte hade haft. Eftersom min vardag bestod av repitition var min impuls att skriva om det jag
såg. Jag hade fattat tycke för de märkliga, bortglömda miljöerna som finns på sjukhus och vårdmottagningar,
som existerade mitt i en plats som aldrig sov, som ständigt var i rörelse, men där miljöerna själva såg
obefolkade och stillastående ut, som att deras funktionalitet hade gått förlorad. En sådan typ av plats som
jag hade reagerat starkt på var de innergårdar, som anlagda med promenadstråk, icke förvuxna buskar och träd
och små konstellationer av parkmöbler övervuxna med alger, som låg insprängda mellan sjukhusbyggnaderna, eller
korridorerna med offentlig konst som var utplacerad genom sjukhuset. Min ursprungliga idé var att tonsätta de
här sjukhusmiljöerna, men miljöerna speglade även min egen tillvaro som, likt dessa platser, kändes
stillastående och tidlös. Projektet började över tid frångå den första idén, för att istället handla om att
dokumentera min egen stillastående vardag.

När jag började må bättre fann jag att det inte hade varit tid utan ork jag hade saknat tidigare, och det var
först då som jag hade orken att ta mig för att börja arbeta med den här idén. I och med en förbättring av
mitt mående och ett överskott av ledig tid kunde jag ta mig an att lära mig programmeringsspråket
SuperCollider, ett språk utvecklat för ljudbehandling och komposition, ett verktyg som jag
tidigare önskat lära mig men inte orkat. I och med detta och ett ökat intresse för datorprogrammering i
allmänhet föll det sig naturligt att mycket av musiken jag skrev blev skriven med hjälp av detta verktyget.

Under min utbildning har det ofta känts som att jag jagat ett ideal av vad elektroakustisk konstmusik är. En
känsla av att inte hinna lära sig alla tekniker som är aktuella inom det elektroakustiska fältet, och att
ständigt göra musik som på ett eller annat sätt är en del av en lärandeprocess, men som är påeldad av
prestationsångest och en känsla av att inte veta vad man gör. En insikt var att mycket, om inte allt jag gjort
under åren varit något likt en ``tech-demo'', ett ljudande exempel på vad en teknik kan erbjuda, utan att ha en
känslomässig förankring till materialet.

Den kompositionsmetod jag använde kändes annorlunda från min erfarenhet från att skriva musik inom
utbildningens kontext. Musiken var inte ett medel för att uppnå målet att bemästra en ny teknik. Istället
fanns en känsla av att kunna få utforska tekniken om ett behov fanns, eller av nyfikenhet, men där ett
estetiskt uttryck fick vara förgrund för processen. 

I uppsatsen om lyssnarbeteenden\footnote{François Delalande (1998) \emph{Music analysis and reception
behaviours: Sommeil by Pierre Henry}, Journal of New Music Research, 27:1-2, 13-66, DOI:
10.1080/09298219808570738} av François Delalande hittade jag ett uttryck som kunde beskriva min metod. Han
beskriver tre olika typer av lyssnarbeteenden, eller analysmetoder, och använder Sommeil av Pierre Henri som
material för hans studie. Delalande definierar tre olika beteenden, det \emph{taxonomiska}, det
\emph{empatiska} och det \emph{figurativiserande}. Det \emph{taxonomiska} påminner om traditionell
musikanalys, att försöka strukturera och klassificera ljud längs en tidslinje. Det \emph{empatiska}
intresserade sig för det känslointryck som musiken gav i stunden, utan att tänka så mycket på form. Det var
det tredje, \emph{figurativiserande} som relaterade till hur jag skrev musik. Den som lyssnar med detta
perspektiv tillåts associera fritt kring de ljud den hör, använda ljuden som scener som beskriver ett
narrativ. Lyssnaren förnimmer narrativet eller scenen genom musiken. 

Delalandes text handlar främst om analys av elektroakustisk musik, och vilka roller olika lyssnare sätter sig
i. Utifrån mitt perspektiv såg jag snarare det \emph{figurativiserande} som en generativ metod som kunde
hjälpa mig i mitt komponerande. Den gav mig något att förhålla mig till, en bild som inte hade någon egen
musikalisk tyngd och därför var öppen för tolkning, men som kunde informera andra tekniska beslut, om klang,
textur och struktur.

% Ett exempel är tekniken Ambisonics\footnote{Roger K. Furness, \emph{"Ambisonics-An Overview,"} Paper 8-024,
% (1990 May), AES}, som är en populär teknik för att arbeta med "rumsligt ljud" (en anglicism av "spatial
% audio", då konceptet är svåröversatt, inte att förväxla med "Apple Spatial Audio"), ett koncept som
% innefattar, men som inte är begränsat till surround-ljud, flerkanalsljud och simuleringar av rum. Ambisonics
% har fått stort genomslag inom utbildningar i elektroakustisk musik, och på skolor runt om Europa har speciella
% konsertsalar byggts för att möta de särskilda kraven som ställs av ambisonics. \linebreak

% /////////

% Teknologier har möjliggjort att vi kan skapa extremt avancerade klanger, som är ovanliga eller omöjliga i
% akustiska rum och resonanser, men hur förvaltar vi det och vart ligger vårt ansvar att göra det mer
% tillgängligt?

% En rimligare fråga är kanske hur elektroakustisk musik blivit trots instutitionernas ingrodda åsikt om absolut
% och programmusik?

% koppla till Joanna Demers text. Hur absolut musik förhärskat inom akademien. Kanske koppla till hur många
% utbildningar som jobbar med ambisonics, hur man bygger "ambisonics-kyrkor" runt om i den akademiska
% elektroakusiska världen och vad det gör för musiken. 

% ////////


% Vrida lite på första stycket, börja personligt, sedan skapa en ram i inledningens andra del, t ex Delalande,
% Schaeffer,  Smalley - morfologiska begrepp och teori kring kategorisering av elektroakustisk musik
% Mark Fisher - kritik mot new public management.


\section{Bakgrund}
  % Som state of the nation - var befinner sig det elektroakustiska fältet just nu? Citat och ev om min egen
  % musik. 

Elektronisk musik har alltid varit intresserat av, och haft ett intimt förhållande till teknologi. Utan nya
upptäckter under 1800-talet och vidare under 1900-talet, hade elektronisk musik som vi hör den idag, inte
varit möjlig. Detta syftar på uppfinningar som radion, trådlös och trådburen elektrisk signal,
magnetband, mikrofoner, datorer och dess språk samt digital signalprocessering\footnote{DSP, Digital Signal
Processing} etc. Studion som instrument 
% bygg ut med Brian Eno bl andra.
har alltid varit central inom elektronisk musik, idag mer än
någonsin, med priset av en persondator som ständigt sjunker som en produkt av en kapplöpning mot ``Moore's
law''\footnote{Moore's Law, En av grundarna för Intel som hävdade att processorer kommer dubbla sin kapacitet
varannat år.}, processeringsförmåga och nya billigare tillverkningsmetoder. Idag kan
en produktionsstudio för hemmabruk rymmas i en dator som ryms på ett enda kretskort, som Raspberry
Pi\footnote{Raspberry Pi [https], \emph{https://www.raspberrypi.com/}}, med ett pris som ligger mellan \$15–35. Tillgängligheten till en studio, som tidigare var begränsad till instutitioner som nationella
radiostationer har gradvis hamnat inom räckhåll för individens händer.

  % Demokratiseringen av studion.
I och med att denna teknik förflyttats närmare individen har den demokratiserats. Musikteknik har i teorin
aldrig varit så tillgängligt. Dessa hjälpmedel för att skriva, spela in och producera kan ibland rymmas i
fickan på sin användare.
% Bygg ut, vad finns det för några hjälpmedel / verktyg idag? Garageband etc
% Bygg ut, vad finns det för några hjälpmedel / verktyg idag? Garageband etc

Trots denna tillgänglighet finns det fortfarande barriärer. Hinder i form av svårtillgänglig information,
eller avskräckande konfigurationssteg som förutsätter en hög grad av förkunskap för att du ska kunna få
tillgång till funktionaliteten du egentligen vill komma åt. Den tidigare nämnda
Raspberry Pi-mikrodatorn stämmer in på den beskrivningen, som trots sitt överkomliga pris förutsätter att du
är bekväm att lämna etablerade operativsystemsparadigmer som Windows och MacOS, kunna utföra kommandon i en
"shell"-miljö, och kunna felsöka när installationer eller konfigurationer inte genomförs som de ska i tidigare
nämnda shell-miljö, eftersom dessa kommandon kan vara mycket obskyra för den oinvigde.
	
  % Skyll på att det är den fria marknadens fel att vi inte använder open source
På detta vis är vi också fast i en marknad där dessa open source-verktyg drunknar på grund av sin egen
otillgänglighet, verktyg\footnote{Jag använder termen "verktyg" för att prata om musikutrustning och
teknologier i en bred bemärkelse, från hårdvara till mjukvara} som annars skulle kunna öppna dörrar för många
till elektronisk musik. Denna marknad består av musikteknologi som säljs för höga summor, något som med open
source-teknologi skulle kunna vara mycket lätt att ersätta, men som marknadsförs till oss som en "easy fix",
något som inte bara underlättar och överbryggar tröskeln till det tekniska, men också till det estetiska.
Dessa verktyg säljs med idén om att du bara behöver skaffa dig denna sista sak, denna pryl, för att få
kreativiteten ett flöda.
Tidigare banbrytade verktygstillverkare, som Moog, eller Dave Smith Instruments, säljer nu sina instrument
om high-end-utrustning. De är brukbara och låter väldigt bra, men de är så dyra att priset stänger dörren för
den som som försöker ge sig in i musiken, eller är demoraliserande för den som tänker att verktyget kommer
lösa dess problem. De är otillgängliga ekonomiskt men ur ett popkulturellt perspektiv mycket tillgängliga,
eftersom populära verktyg blir synonyma med ett ljud, en stil, en kultur och med musikalisk och ekonomisk
framgång. Ett exempel på detta är den legendariska TB303\footnote{TB303}-synten från Roland, som genom sitt
stilbildande ljud som idag är synonymt med elektronisk dansmusik, som idag är värld säljs för mellan 2 - 5
gånger sitt säljpris, år 1982 (Originalpris: \$395.00 . Efter inflationskompensation: \$1,161.32. På Ebay
2022: \$5,543.64).
Det här fenomenet kallas för \emph{G.A.S.} - \emph{Gear acquisision syndrome} (eller på svenska
\emph{utrustningsanskaffningssyndrom}), och yttrar sig i att man fullständigt accepterat det narrativ som säljs
till en. Om ett företag säljer ett nytt verktyg, med en effektiv marknadsföring bakom sig, kommer forum och
trådar på sociala medier fyllas med ivriga entusiaster som proklammerar att de \emph{"behöver detta!"}. Inom
marknaden för hårdvara, som synthar och studioutrustning, är det här en stark köpargrupp, och det går
inflation i instrument som är inovativa. Istället säljs i hög grad verktyg som är kopior av tidigare erkända
verktyg. Ett tydligt exempel på detta är Behringer, som till synes baserat sin affärsmodell på kraften hos
\emph{G.A.S.}, då de släpper kopior av intrument som tidigare varit ouppnåeliga, och gör det på så sätt möjligt
för många med \emph{utrustningsanskaffningssyndrom} att "anskaffa" den utrustning de känt att de saknat. 

  % otillgängligt för majoriteten av musikutövare, och kan istället existera som en statussymbol mer än ett
  % kreativt verktyg. 

  % vänd på det och säg att inom EAM så känns det ofta som man har en motsatt inställning, 
  % är motståndet att vi förkastar konventionella teknologier för svårtillgänglig, och har det då varit
  % skadligt för EAM i stort?

Inom elektroakustisk musik kan snarare en motsatt reaktion uppstå. Här ges de mer obskyra verktygen spelrum,
och blir i sig en statussymbol i sin otillgänglighet. De tidigare nämnda verktygen som är svåra för den
vanliga brukaren av musikteknologi blir här ett verktyg för att visa sin virtuositet. 
är datorprogrammering. 
Istället för att använda färdiga verktyg för att åstadkomma ljud, och applicera sin kunskap om olika
syntesmetoder på en mjuk- eller hårdvarusynt, kan du nu skriva ljudprocesseringsalgoritmer själv och på egen
hand hushålla med din dators resurser. Visserligen möjliggör sådan teknik större flexibilitet och en
inspirationkälla i att utforska tidigare onåbara delar av ett ljudprogram, icke-linjära strukturer av musik
och möjligheten att påverka ljud i ett "close to the metal"-perspektiv, bearbetningar av dess data och form
nere på en processorcykelnivå. Här kan \emph{G.A.S.} yttra sig annorlunda, framförallt inom en utbildningar av
elektroakustisk musik. I podcasten \emph{La Meme Young}\footnote{La Meme Young, \emph{THE JESSICA EKOMANE
TALK}, 06-12-2021} har Max Alper och Jessica Ekomane ett samtal om att
fastna för mycket i den tekniska aspekten av elektronisk konstmusik. De beskriver ett scenario där en student
på masternivå arbetar under hela sin studietid med att bygga sitt verktyg, sitt instrument, förfinar alla
aspekter och reder
ut alla buggar och grafiska fulheter, för att komma ur utbildningen utan att ha skrivit
musiken det här instrumentet skulle hjälpa till med att skriva. Hur instrumentet i sig blivit så omfattande
och stort att det är svåröverskådligt hur ett komponerande med det ens skulle se ut. 
Joanna Demers skriver i "Listening Through The Noise": \emph{
The sheer freedom of electroacoustic music constitutes both its strength and its burden. \footnote{Joanna Demers, \emph{Listening Through The Noise, The Aesthetic of Experimental Music}, New York och Oxford: Oxford University Press, 2010}}.
Joanna Demers pratar om den oslagbara friheten hos elektronisk musik. Klangvärlden hos elektronisk musik har
till synes oändlig potential, som trots vad detta utlovar, kan vara till dess nackdel. I Linus Hillborgs uppsats
Återbruk och återgivning\footnote{Linus Hillborg, \emph{Återbruk och återgivning}, 2021}, skriver han om teknologiers begränsingar, hur det kreativa
kan komma ur möjligheternas begränsning. Genom att utforska var den gränsen går kan man även hitta de
särskiljande kvalitéer som möjliggör något som är unikt för den teknologin. Det kreativa arbetet kan då likas
vid att känna sig fram i blindo längs en ojämn yta för att lokalisera och kartlägga det intressanta, samt att
lyfta fram det i ljuset. 


% DIY

% Donna Haraway - Cybernetic Manifesto

% Brian Eno - studion som instrument. 

\subsection{Absolute vs Programmatic music.}
I sin text ``Listening Through the Noise'' diskuterar Joanna Demers hur man som elektroakustisk kompositör kan
förhålla sig till extramusikaliska referenser. Hon hävdar att inom konservatorier har ``absolut musik''
historiskt sett fått störst spelrum och har ansetts som norm. Det är en modernistisk stil som speglar den
modernistiska genikulten, där musiken eller konsten i stort uppstår ur geniets sinne, gärna frånkopplat från
yttre störningsmoment. Likaså är den ``absoluta'' musiken en musik vars mening och syfte är sprunget ur dess
egen genialitet, istället för att hänvisa eller berätta om något extramusikaliskt.

Med extramusikaliskt menas något som ligger utanför musiken, ``lying outside the province of
music''\footnote{Merriam-Webster, \emph{https://www.merriam-webster.com/dictionary/extramusical} (Hämtad
29-03-2022)}, strikt tolkat allt som inte innefattas inom det fysikaliska i ljudvågor som sammanfaller i
luften, eller som innefattas inom de regler för ljud-, form- och harmonilära används för att beskriva musik. 

Idag benämns motsatsen till den ``absoluta musiken'' som ``programmusik''. Det härstammar från när
kompositörer skrev musik som ville gestalta något extramusikaliskt, där musiken var det bärande berättande elementet, där programbladet på en sådan konsert var en viktig bilaga
för att förstå handlingen i musiken. 

Demers frågar sig var den elektroakustiska musiken platsar in mellan dessa två motstridiga begrepp och
benämner två kompositörer som har varit viktiga för teoretiserandet kring elektroakustisk musik, \emph{Pierre
Schaeffer} och \emph{Trevor Wishart}. Hon menar även att den elektroakustiska musiken andra förutsättningar än
traditionell, akustisk musik, på grund utav dess ljudliga frihet och frågar sig
vad det innebär för dess associativa förmåga. ``
\emph{Are sounds always referential? Is it possible to hear sound before it has been laden with the
associations of culture, history, or society?}
\footnote{Joanna Demers, \emph{Listening Through The Noise, The Aesthetic of Experimental Music} s. 23, New
York och Oxford: Oxford University Press, 2010}''.

%				etymologin för "akusmatisk"
%
% Chapter 4 of Pierre Schaeffer's Traite des Objets Musicaux (Schaeffer 1966)
% is entitled The Acousmatic. According to the definition in Larousse, the
% Acousmatics were initiates in the Pythagorean brotherhood, who were
% required to listen, in silence, to lectures delivered from behind a curtain such
% that the lecturer could not be seen. The adjective acousmatic thus refers to
% the apprehension of a sound without relation to its source.

Kan de ljud som frammanas inom elektroakustisk och akusmatisk musik existera utanför sig
själva, vara självrefererande, eller om de kan existera, likt Pierre Schaeffers teori om
ljudobjekt\footnote{Pierre Schaeffer, \emph{Traité des Objets Musicaux} Paris: Édition du Seuil, 1966}, som
ett ljud där ursprunget inte längre är urskiljbart? 


Begreppet akusmatisk härstammar från Pythagoras\footnote{Oxford English Dictionary, \emph{Acousmatic (adjective)}, https://www.lexico.com/definition/acousmatic, (Hämtad 04-04-2022)} och används för att beskriva
ett ljud vars källa inte är synlig. Det har inom den elektroakustiska traditionen använts för att beskriva de
teorier Schaeffer lade fram, kring att analysera och tolka ljud utan att se objektet som skapar ljudet.
Det ska då gå att tolka ljudet utifrån dess texturer och kvalitéer snarare än dess förmåga att berätta något om dess
källa. Schaeffer myntade uttrycket ``reducerat lyssnande'' för att beskriva denna analysmetod, som ska göra
lyssnaren fri från ljudets inbyggda associationer. Istället ska man se ljudet som det existerar i sin mest
avskalade, ``rena'' form. Demers ställer Schaeffers teori kring ljudobjekt mot Trevor Wisharts omtolkning av
begreppet. Demers skriver: ``\emph{Wishart feels that it is impossible to separate sounds from their
associations, so it is incumbent upon the composer to acknowledge and work with sound references rather than
repress them}''. I On Sonic Art\footnote{Trevor Wishart (1996), \emph{On Sonic Art}, Amsterdam:
Harwood} ställer sig Wishart skeptisk till Schaeffers reducerade lyssnade. Wishart menar att ljud aldrig kan
vara frånskiljda en association.
%, och lägger istället bedömningen om ett ljuds associativa, extramusikaliska egenskaper hos
%lyssnaren, mottagaren av det musikaliska meddelandet. 
Wishart använder sig istället av begrepp som reella och
icke-reella ljud, reella och icke-reella rum samt något som han benämner som imaginärt/surrealistiskt, för att
prata om musik som landskap\footnote{Ibid, Kap. 7: Sound landscape}. Det han menar med reella och icke-reella
är att de kan vara naturligt förekommande (röst, traditionella musikinstrument eller naturen etc.) eller från
en onaturlig, konstgjord källa (ett elektroniskt bearbetat ljud eller synthesizer etc), som sedan placeras i
en naturlig, icke konstgjord miljö (ljudets egen resonanta klang från rummet som fastnade på inspelningen)
eller en konstgjord miljö (artificiella rumsklanger eller onaturliga rumsliga effekter), som sedan samspelar
för att skapa en ljudande helhet som han liknar vid ett landskap, som beroende på hur man använder sig av de
olika typerna av ljud han beskriver, kan verka verklig eller overklig. I hans resonemang är det mer intressant
att bejaka den assiciativa förmågan hos ljud, att det är nödvändigt för kompositören att ta hänsyn till ljudens förmåga att referera för att få ett starkare grepp över det man som kompositör vill
förmedla. 

Det akusmatiska lyssnandet, likt den absoluta musiiken, bygger på att musik och ljud kan ha ett värde i sig
själv. 
På samma sätt är programmusik besläktat med det associativa, refererande lyssnandet, där ljudets berättande
förmåga får stå i fokus, ett arbete som delvis även överlämnas till lyssnarens tolkning.

% Russolo - fokus och intresset kring krigsljud och krigsmaskiner, något som ligger till grund för tidig eam,
% som är manligt kodat och fascistiskt.

% I futuristiska manifestet\footnote{futuristiska manifestet}, skriver Filippo Tommaso Marinetti om den nya
% konsten som ska upphöja och hylla det nya, motorer och hastigheten i industri och urbanisering, samt till
% kriget, som är "världens enda hygien". Det är aggresivt misogynt och antifeministiskt, vilket stämmer in på
% dess koppling till fascismen och dess dyrkan av styrka och militarism, samt dåtidens kvinnosyn.
% Luigi Russolo, som var en kompositör inom den futuristiska rörelsen, ansåg att det var rättighet och en
% skyldighet att skriva musik som okunnig. Han skrev i Oljudets konst\footnote{Oljudets konst, Lárte dei
% rumori} att det var först då vi kunde komma undan akademien och etablissemangets hårda grepp runt vilken
% musik som fick tillverkas. Att även man skulle sluta imitera gamla tonsättare och tekniker och göra dagens
% musik.\linebreak


% Detta är den av de få
% punkter från futuristerna manifest som kan ha någon vikt idag. Dock lyder den fullständiga formuleringen
% (punkt 10 i manifestet):


% "Vi vill förstöra museerna, biblioteken, akademier, av alla slag och bekämpa moralism, feminism och varje
% opportunistisk eller utilitaristisk feghet."


% Luigi Russolo lät bygga ett antal maskiner, som skulle reproducera "stadens musik" i en
% konsertsättning. Dessa kan ses som ett tidigt exempel på elektroakustisk musik, men också på musik som ville
% bryta mot den, som Joanna Demers beskriver, förhärskande "absoluta" musiken som lärdes ut på
% konservatorier. Futuristerna ville på ett väldigt bokstavligt sätt få sin musik att handla om något utanför
% sig själv. 


% Med detta exempel vill jag prata om hur elektroakustisk musik i sitt ursprung ofta tagit ett uttryck i
% programmusik, och även ett avständstagande från den absoluta musiken. 

% Russolo och futuristerna på början av 1900talet var intersserade av maskiner, industri och krig, med tydliga
% fascistiska konnotationer. Deras arv kan fortfarande höras inom den elektroakusiska genren, med sitt
% intresse för ljud som kommit som biprodukt ur industri och från maskiner, och stundtals rent aggressiva
% ljud, som t ex noise. 

\subsection{\emph{Liminality} och \emph{The Eerie}}
Liminalitet är ett begrepp som föddes ur antropologin och myntades av Arnold van Gennep i hans bok \emph{Rites
of Passage} (först utgiven 1909) Uttrycket kom till under hans forskning på ritualer, för att kunna beskriva
gemensamma mönster som vanligtvis återfinns i ritualer från alla kulturer, där det \emph{liminala}
kännetecknar ett mellantillstånd i ritualen. Han beskriver två ytterligare stadier, en symbolisk (eller i
vissa fall verklig) död och pånyttfödelse, ett före och ett efter, där det ``liminala'' är
tröskeln mellan de två, där förändringen sker.
Gennep skriver:

\begin{quote}
I propose to call the rites of separation from a previous world, preliminal rites, those executed during the
transitional stage 'liminal (or threshold) rites, and the ceremonies of incorporation into the new world
postliminal rites.\footnote{Arnold van Gennep, \emph{The Rites of Passage}: s. 21,  Chicago: The
University of Chicago Press, (1977)}
\end{quote}

% \begin{itemize}
% 	\item preliminal rites (or rites of separation): This stage involves a metaphorical "death", as the
% 		initiate is forced to leave something behind by breaking with previous practices and routines.
% 	\item liminal rites (or transition rites): Two characteristics are essential to these rites. First, the
% 		rite "must follow a strictly prescribed sequence, where everybody knows what to do and how". Second,
% 		everything must be done "under the authority of a master of ceremonies". The destructive nature of
% 		this rite allows for considerable changes to be made to the identity of the initiate. This middle
% 		stage (when the transition takes place) "implies an actual passing through the threshold that marks
% 		the boundary between two phases, and the term 'liminality' was introduced in order to characterize
% 		this passage."
% 	\item postliminal rites (or rites of incorporation): During this stage, the initiand is re-incorporated
% 		into society with a new identity, as a "new" being.\footnote{Arnold van Gennep (1977), \emph{The Rites
% 		of Passage} s. 21, Chicago: The University of Chicago Press}
% \end{itemize}

I texten \emph{Liminality and Experience} beskriver Arpad Szakolczai etymologin av ordet liminalitet, som
sprunget latinska ordet limen/limit, som betyder tröskel eller gräns\footnote{Arpad Szakolczai,
\emph{Liminality and Experience: Structuring transistory situations and transformative events}, International
Political Anthropology Vol. 2, No. 1, s. 147-148, 2009}. Han förtydligar begreppen pre- och postliminalitet,
som en metaforisk död, separationen från sitt tidigare liv, och en pånyttfödsel, där man återförs till världen
förändrad av ritualen. Ett exempel som Gennep tar upp är ritualen som återfinns i många kulturer kring att bli
vuxen. Oavsett hur detaljerna kring en sådan ritual ser ut återfinns nästan alltid strukturen av ett stadie av
separation från världen och sin nuvarande form och ett stadie av återinträdande i världen pånyttfödd i en ny
form. I den här strukturen finns även tröskelstadiet, det liminala, där förändringen från barn till vuxen
sker. 

Det har funnits flera försök att applicera begreppet liminalitet på något utanför ritualen, att låna modellen
för att beskriva andra skeenden. Ett sådant exempel är medeltiden, som på flera sätt var en tillbakagång i
Europas vetenskapliga utveckling. Under medeltiden föll många framsteg inom vetenskap och medicin i glömska,
och vissa återuppfanns inte förrän under industrialiseringen. Man kan med begreppet liminalitet beskriva delar
av historien där utvecklingen inte varit konstant, där den avstannat, gått tillbaka, för att sedan återhämta
sig. 

Ett exempel som beskriver en liminal plats finns i Mark Fishers bok ``The Weird and the Eerie''. I hans text
omtolkar och delar upp det freudianska begreppet \emph{the Strange}, ``det märkliga'', till \emph{the
Weird} och \emph{the Eerie}, där det sistnämnda uttrycket kan användas för att beskriva känslan av en liminal plats. 
\emph{The Weird} kännetecknas av en närvaro av något som känns fel, som inte hör hemma. Ofta en känsla som
infaller när man upplever något nytt och oigenkännbart. \emph{The Eerie} är istället en märklighet som undergräver ens
förväntningar. Det är en känsla som uppstår vid avsaknaden av närvaro, eller en närvaror där vi väntar oss
den. Han citerar sin audio essay ``On Vanishing Land: M. R. James and Eno'' i kapitlet med samma namn, där han beskriver en
promenad längs en kuststad i Storbrittanien som ett exempel på \emph{the Eerie}.

\begin{quote}
The port and the burial ground offer two different versions of the eerie. The container port looms over the
declining seaside town, the ports cranes towering above the Victorian resort like H.G. Wells' Martian Tripods.
Approached from the countryside, from Trimley marshes, the cranes preside over the rural scene like gleaming
cybernetic dinosaurs erupting out of a Constable landscape.\footnote{Mark Fisher (2016), \emph{The Weird and the
Eerie}, kap. On Vanishing Land: M.R. James and Eno, s. 76, London: Repeater Books}
\end{quote}

Han beskriver känslan från att se hamnen, vars verksamhet flyttats till en större, mer centraliserad hamn. 
Hur både staden och hamnen som förväntats vara fylld av människor och rörelse nu är motsatsen, och hur kuslig
den är på grund det.

I samma kapitel diskuterar Fisher Brian Eno, vars album \emph{Ambient 4: On Land} handlar om den del av
Storbrittanien som han själv skriver om. Eno var född i Suffolk och albumet är ett försök till att skriva
musik om landskapen där han växt upp. Fisher skriver: 

\begin{quote}
The shift into sound opens up the eerie. There is an intrinsically eerie dimension to acousmatic sound — sound
that is detached from a visible source — and one of the most unsettling tracks on On Land is "Shadow”, which
features a quietly distressing whimper that could be a human voice, an animal sobbing, or an aural
	hallucination produced by the movement of wind.\footnote{Ibid, s. 81}
\end{quote}

Han sammankopplar här det \emph{liminala / the Eerie} med det akusmatiska, och definierar det akusmatiska,
reducerade lyssnandet som närvaron hos ljudets källa som uteblir. Han nämner spåret "Shadow" som särskilt
\emph{Eerie} och att det framkallar ett antal associationer hos honom när han tolkar de akusmatiska ljuden.
Känslan av \emph{the Eerie} kan då framställas av det akusmatiska, en avsaknad av kontext, och denna avsaknad
öppnar för ytterligare association hos lyssnaren.

% Junk Space - Rem Koolhaas - tuffa arkitekten som Malte nämnde)
% Mark Fisher - The Wierd and the Eerie 
% Brian Eno - Music For airports


%   Robin James - Resilience and Melancholy

  % Edu-art - konst som är producerad inom akademin, som inte har något värde utanför den världen.

  % som konstnärer behöver man inte ge ifrån sig något som är bra.
  % som forskare så kan man inte tweaka 

  % Kölnstudion
  % Parisstudion

  % I came into the studio to make noises speak, I stumbled into music - Pierre Schaeffer

  % Henk Borg - Frailing

  % on, for or in the arts: 
	% on: musikvetenskap, musikhistoria - forska på musiken själv
	% for: barockinterpretation - syftet är att spela barock
	% in: 

	% konstnärlig forskning vad är det?

	% Henri Bergson - Introduction to metaphysics

  % Denis Smalley - om division of labor

  % Simon Emmerson - The relation of language to materials

  % Ens subjekt, ens röst, som man odlar, kontra t ex AI-musik, där du kodar bort ditt eget subjekt
  % I want to hear someones music, not someones max-patch. i AI är det kanske omvänt
  % AI är kanske den utlimata kritiken mot det jag kritiserar. 

  % Åsa Stjerna - Before Sound - om att släppa kontrollen om musiken 
  % Cybernetic manifesto

\subsection{Teknik som inspiration}
- Att inspireras av teknik, dess möjligheter och dess begränsningar. Detta är en viktig rubrik, då det är ett
  en viktig del av elektronisk musik, och en viktig del av mångas metod. 

  Linus Hillborgs examensarbete. exemplet med metal gear solid

\subsection{Tekniken som en börda}
G.A.S. - gear acquisision syndrome
- Arbetssätt och teknologi som en merit, där ju otillgängligare något är, eller mer obskyrt, desto större vikt
  har det inom elektronisk musikkretsar. 

  Konceptet gate-keeping, att hålla på "the fruits of your labor" istället för att dela kunskap, för att ha
  ett övertag över ens jämlikar.

  Maria Horns examensarbete - Om manligt och kvinnligt kodade egenskaper, och hur tekniken varit något som män
  länge fått höra att de har rätt att använda, för de har rätt att kontrollera världen, medan kvinnor inte har
  det. Detta har bidragit till att kvinnor inte har lika lätt in i rum som dominerats av män, men en
  tekno-jargong som passivt (eller aktivt) exkluderar kvinnor. 

  Fokuset på att bygga sitt verktyg, och att man kan gå vilse i den processen. Jean Baudrillard, högre
  upplösning av mediet, lägre upplösning av innehållet(?). 
  La Meme Young-podcasten med Jessica Ekomane, där de pratar om den hypotetiska masterstudenten som spenderar
  2 år att bygga sin enorma \emph{patch}, som ska lösa alla problem och äntligen möjliggöra den där musiken
  som hen alltid drömt om, för att sedan gå vilse i syftet och tappa siktet på slutmålet, att göra musik. 

  % t ex stycken som brukar extended techniques, men utan att veta varför
  % Buchla - "we are not limited by our tools but our mindsets"

  % vad är antitesen till "megapatchen"? att aldrig spara något? att göra något i stunden (live-kodning)? 
  % en skillnad mellan oss och andra instrumentalister.


\section{Koppling till musiken}
- Koppla till min musik, till EPn som jag gjorde som start för examensprojektet, och även de experiment som
  jag gjorde med ljud på sample-nivå. 
- Började med en idé om att tonsätta och dokumentera platser, de jag såg under min sjukskrivning, framförallt
  de övergivna offentliga rum som fanns i de miljöerna. Hur det projektet övergick till att dokumenter min
  egen vardag, som på något sätt länkades ihop med de platserna. Beskriv hur begreppet
  liminalitet\footnote{liminalitet} kan användas för att beskriva båda dessa, då det beskriver att vara i
  tröskeln av någonting. En tröskelplats, en plats på vägen mellan två platser, eller en plats i en
  övergångsperiod, på väg till att bli något annat. Eller ett tröskeltillstånd, som i ritualen, där det finns
  ett före och ett efter, men under ritualen befinner man sig i ett mellanläge. Hur min situation som
  sjukskriven och bunden till mitt hem kunde ses som ett sådant tröskeltillstånd.

  Den ljudande delen av mitt examensarbete består av en EP släppt på kassett på bolaget
  Kalkatraz\footnote{länk till kalkatraz nånting}, samt en ljudinstallation, som i en omarbetning även blev
  ett stycke för piano och elektronik.

  Min EP skrevs i huvudsak under min tid som sjukskriven efter min njurtransplantation. De fyra spåren har
  namn som refererar till tidpunkter eller platser som fastnat tydligt i minnet hos mig. 
	\subsection{DEC}
	\emph{Dec} var det första stycket jag började arbeta med. Ursprungligen var detta ett stycke som skulle
	vara en form av ljudläggning av platser på sjukhus ( se inledningen ) som känns bortglömda, med fokus på
	innergårdar och utrymmen runtomkring och mellan husen, ofta med ett gråaktigt, deprimerande eller illa
	skött offentlig konst. En komposition om att spendera mycket tid i dessa miljöer. Dock övergick projektet
	efter hand till att beskriva den första tiden då jag var sjukskriven, när jag abrupt pausade mina studier
	och påbörjade en tillvaro som var ett typ av mellantillstånd, där varje dag smälte ihop i nästa, avbrutna
	av läkarbesök.
		Jag började arbetet med att fundera över signal- och styrflöde i ett modulärt system jag nyligen hade
		byggt. Jag hade en plan om att kunna styra synten med MIDI\footnote{MIDI} från SuperCollider, och att
		den sedan skulle efterarbetas i Reaper:
		\begin{center}
			% tabular behöver ytterligare argument som visar hur många celler 'c'
			\begin{tabular}{ c c c c c }
				% avgränsningen görs med '&' mellan celler och '\\' mellan rader
				Midi		  & -> & Ljud			& -> & Processering/mastring \\
				SuperCollider & -> & Eurorack-synth & -> & Reaper
			\end{tabular}
		\end{center}

		Till en början var arbetet att få en lösning med MIDI för att skicka ett playback av tonhöjd- och
		durationsmaterial till synten. Detta var något nytt för mig och jag fick det inte att fungera som jag
		ville (inte alls), så jag övergav idén för intern playback i synten själv. 
		Jag arbetade med ett QPAS-filter (Quad Peak Animation System) från tillverkaren Make
		Noise\footnote{makenoise}. Med detta filter kunde jag arbeta med ytterligare en stämma, genom att
		framhäva och finstämma vissa övertoner i den styckets klang. Stämman bidrog till variation i en
		annars väldigt lunkande och stillastående ljudbild, Jag upplevde att den bidrog till mer motrörelse,
		trots att stämman var en en ton med en statisk relation till de underliggande tonerna. Med denna
		parameter samt möjligheten att bredda och öppna filtret för att släppa igenom mer ljud, och varsin
		wavefold-effekt på oscillatorerna, hade jag möjlighet att framföra stycket med ett mått
		"liveness"\footnote{citat om liveness i elektronisk musik}. Ytterligare hade jag en bruskälla och ett
		LFO\footnote{LFO} som gick ut från från SuperCollider till synten, för att tillföra ett mått av slump.
		Jag kunde styra den från samma kontroll som de andra. Detta kändes viktigt då jag tanken var att
		spela musiken som en solokonsert, och med tanken på min bakgrund som pianist kändes det rätt att ha
		något att påverka musiken med. 

	\subsection{Koltrast}
	Stycket \emph{Koltrast} skrevs under sommaren 2021, när jag spenderade tid i Göteborg. Stommen i stycket
	är en lång field-recording av koltrastsång. Efter min operation hade jag haft problem med att sova, och
	inspelningen gjordes under en natt som var så varm att det var tvunget att ha balkongdörren öppen. Precis
	när det började ljusna vaknade jag av att det lät som fåglar inuti rummet. Det visade sig vara en handfull
	koltrastar som sjöng över hustaken på Doktor Fries Torg. Fångade det genom att plocka fram en
	Zoom\footnote{portastudio Zoom} innan jag gick och lade mig igen. När jag sedan bestämde mig för att
	skriva ett stycke kring fågelsången, ville jag testa om jag kunde låta inspelningen styra musikaliska
	parametrar på något vis. Min första tanke var att använda FFT-analys\footnote{FFT-analys} för att kunna
	följa fågelsångens egen tonhöjd, och tillåta den styra tonhöjden av en synt. Under processen hade jag blev
	jag dock mycket fäst vid hur inspelningen lät i sig själv och ville inte bygga ett instrument som skulle
	göra en konstgjord kopia, om än möjligtvis intressant. Istället bestämde jag mig för att bygga ett program
	i SuperCollider som kunde följa amplituden av fågelsången, och ta beslut kring musikaliska händelser när
	amplituden översteg ett tröskelvärde. Synten som triggades av dessa event hade en tydlig attack, men om
	dess eget volymenvelop öppnades upp och tillät för längre toner påminde det om inspelningar jag hört av
	orgelpipor, där mikrofonen stått mycket nära mekanik och klaffar. Jag bestämde mig då för att det skulle
	bli något i stil med ett syntetiskt orgelstycke.

	%%%% SÄTT IN LITE SC-KOD HÄR %%%%

	Jag undersökte även ``physical modelling'', tekniken att framställa ljud som liknar akustiska klanger på
	syntetisk väg. Till detta använde jag en färdig klass i SuperCollider som heter DWGBowedTor från
	biblioteket DWG\footnote{DWG}. Tillsammans med en basstämma fick agera kontrapunktiskt mot den mer
	stokastiska koltraststämman och dess ackompanjerande synt, de fick vara mer förutsägbara och ha en mer
	cirkulär form. Här arbetade jag även med att skriva sekvenser med ``kontrollstrukturer''\footnote{SC
	controll structures}, och göra enkla sekvenser som varierade sig baserat på vilket varv vi var på i
	loopen. 
	\pagebreak

\renewcommand{\baselinestretch}{1}
\begin{lstlisting}[style=SuperCollider-IDE, caption=Amplitudtriggad synt]
SynthDef(\demandSynth2, {
	var fund = 69;
	var trigger, in, sig, env, fade, verb, fadeOut;

	in = A2K.kr(In.ar(\in.kr(55), 2) * 4);
	fade = Line.kr( 0, 1, \fadeTime.kr(2));
	trigger = Trig.kr(			    // <-- Här sker triggningen
		InRange.kr( 
			Amplitude.kr(
				A2K.kr(in.lag(1) * 4), 0.2, 0.5
			), \loTresh.kr(0.01), 0.8 
		).lag(0.01)
	);

	env = EnvGen.kr(
		Env( 
			[ 0, 0.01, 1, 0 ],
			[
				0.05,
				\atk.kr(0.1).linexp(0, 1, 0.01, 0.3),
				\rel.kr(1).linexp(0, 1, 0.01, 0.39) 
			], curve: -4
		), trigger
	);

	sig = DPW3Tri.ar( Demand.kr(trigger, 0, demandUGens: Dseq( [ 
			[ fund, (fund*3) / 2], // Db, Ab
			[ (fund*5) / 4, (fund*5) / 3], // F, Bb
			[ (fund*15) / 8, (fund*9) / 8], // C, Eb
			[ (fund*3) / 2, (fund*45) / 32],  // Ab, G
			[ (fund*15) / 16, (fund*5) / 4]
			] * 2, inf
		) * Dwrand([1, 2, 0.75], [0.6,0.2, 0.1], inf)).lag(0.1)
	);

	sig = sig * fade * EnvFollow.kr( in ).lag(0.7) * \vol.kr(0.7) ;
	sig = HPF.ar(sig, \hpf.kr(440));
	fadeOut = Line.kr(1,0,\fadeTime.kr);
	verb = NHHall.ar(
		sig, 
		\verbTime.kr(4).linlin(
			0, 1, 0, 12
			), 
		0.5, 800, 0.5, 2000, 0.2, 0.2, 0.3
	);

	Out.ar( \out.kr(0), 0.8 * env * sig!2.tanh + ( 
		verb * \verbVol.kr(0.3).linexp(0, 1, 0.01, 0.6))
	);  
}).add;
\end{lstlisting}
\renewcommand{\baselinestretch}{1.5}


  I exemplet ovan visar jag min lösning på hur jag triggade synten på amplitud (på rad 7), samt nedan hur jag
  hur jag använde många styrparametrar i varje ljud för att kunna påverka ljuden i realtid. Objektet ``f''
  representerar min midikontroller, och ``f.faderAt(1)'' motsvarar fadern på plats 1.


\renewcommand{\baselinestretch}{1}
\begin{lstlisting}[style=SuperCollider-IDE, caption=Syntens kontrollschema]
~follow2 = Synth(\demandSynth2, 
	[ \loThresh, f.faderAt(1).asMap, // asMap, annars används bus-numret som värde,
		\atk, f.faderAt(2).asMap,		   // kan bli apstarkt!
		\rel, f.faderAt(3).asMap,
		\verbTime, f.faderAt(4).asMap,
		\verbVol, f.faderAt(5).asMap,
		\vol, f.faderAt(6).asMap,
		\hpf, 350],
		addAction: \addToTail); 
\end{lstlisting}
\renewcommand{\baselinestretch}{1.5}

  \subsection{K87-89}
  K87-89 heter sjukhusavdelningen på Huddinge sjukhus där jag och min pappa var efter min operation. Den här
  perioden, och de kommande veckorna präglades av en väldigt hög energinivå hos mig själv, delvis på grund av
  ett mycket förbättrat fysiskt mående, men också definitivt på grund av de starka mediciner jag fick.
  Kopplingen till det liminala var att vi, mitt under Covid19-pandemi, var tvungna att stanna på sjukhus en
  vecka, med bara varandra och sjukhuspersonal som sällskap. Detta i kombination med hur en operation är en
  slags ritual, en exorcism av det onda. 

  Jag ville fånga denna post-ritualistiska extas av febrig energi i ljud, vilket slutligen resulterade i detta
  stycke. Eftersom jag har en förkärlek för långsamma klanger var min första tanke inte att göra något
  rytmiskt intensivt. Jag började istället att försöka skapa ett ljud som kunde gestalta ett sprak av
  elektrisitet, som konstant överladdning av synapserna, något böljande. Ett ljud som var ostadigt och
  lynnigt. 

  Ljudet jag kom fram till hittade jag delvis av en slump. Jag experimenterade med ett objekt i
  SuperCollider vid namn ``Fold'', som tog vågformer över en viss amplitud och vände dem inåt mot sig själva
  180° ( se bild ). Det tar en signal som input, samt två värden för tröskelvärdet på den positiva samt
  negativa polen av signalen. Ljudet blev till när jag av misstag satte en av parametrarna till '0', vilket
  vände allt ljud in på sig själv i oändlighet, och ena polen av ljudet blev i praktiken bortklippt. Det
  intressanta med detta var att under vissa omständigheter kunde ljudet få en DC offset\footnote{DC offset}
  och lägga sig i den bortklippta polen, vilket resulterade i fullständigt bortfall av klang, för att sedan
  bryta igenom som från bakom en vägg. Jag rättade till mitt miss men effekten fanns kvar, dock inte lika
  markant, men framträdde mer när jag när jag hade flera i följd, när ljudet passerade genom effektkedjan.

\renewcommand{\baselinestretch}{1}
\begin{lstlisting}[style=SuperCollider-IDE, caption=Fold-synt]
SynthDef(\fold, {  // \t_trig, \freq, \fold.
	var sig, env, out, verb, dirt;

	env = EnvGen.kr(
		envelope: Env(
			[0,1,0], [\atk.kr(0.4), \rel.kr(2.6)], curve: \lin
			), 
		gate: \t_trig.kr(0), levelScale: 1, timeScale: 1, doneAction: 2);

	dirt = LFDNoise3.kr(25);

	sig = LPF.ar( 
		Fold.ar(
			XFade2.ar(
				SinOsc.ar(\freq.kr(64).midicps), 
				DPW3Tri.ar(\freq.kr.midicps), \fold.kr(0.1).linlin(0, 1, -1, 1) + dirt.linlin(0, 1, 0, 0.1)
			),
			-1 * (\fold.kr + dirt.linexp(0, 1, 0.01, 0.05)),
			\fold.kr + dirt.linexp(0, 1, 0.01, 0.05)
		) * env, 5000 );

	Out.ar(\out.kr(0), Splay.ar(sig*\vol.kr(0.1).linlin(0, 1, 0, 0.35))).tanh; // maybe over-did it with .tanh?
	Out.ar(\out.kr+2, Splay.ar(sig!2*\vol.kr).linlin(0, 1, 0, 0.35)).tanh;

}).add;
\end{lstlisting}
\renewcommand{\baselinestretch}{1.5}

  \subsection{Metamorfos}
  Under min sjukskrivning hade jag fått uppmaningen att gå promenader, och jag passade på att spela in ljud i
  min omgivning. Dessa ljud använde jag senare i ljudinstallationen \emph{Metamorfos}, som ställdes ut en helg i
  februari 2022 på Galleri Resorb i Stockholm. Ljudinstallationen hade börjat som ett experiment på
  filstrukturen hos en ljudfil, om det skulle vara möjligt att göra bearbetningar på den som skulle vara
  intressanta. 

  Mitt första experiment var att skriva en ``disintegration loop''\footnote{William Basinsky} (kända
  exempel är William Basinskys \emph{Disintegration Loops}, eller Alvin Luciers \emph{I am sitting in a room}), som
  tömde datastrukturen på innehåll, nollställde varje sample en och en. Effekten blev att jag långsamt
  tillintetgjorde det inspelade ljudet i filen. Transienter hos ljudet försvann och ljudfilen fick en mer och
  mer homogen textur.

  Jag testade sedan samma teknik, men istället för att ``tömma'' en ljudfil använde jag metoden för att göra en
  övergång mellan två ljudfiler, en gradvis transformation från en fil till en annan. Jag fann dock att
  transformationen var så gradvis att när processen var halvvägs mellan den ena och den andra filen var dess
  ljud svårt att åtskilja från vitt brus. Min slutgiltiga version tog därför större bitar, ``chunks'' av
  kontinuerliga samples, där ljudfilerna gjorde hårda klipp in i varann. Jag fann att med ett antal ljudfiler
  som blandades in i varandra kunde en effekt uppstå där det kändes som ljudfilerna spelades parallellt, trots
  att det var omöjligt med tanke på hur programmet var uppbyggt. Vad som hände var en psykoakustisk effekt som
  jag fann vara mycket intressant. Distributionen av dessa ``chunks'' var slumpmässig och beslut togs om
  distributionen efter varje genomspelning av ljudfilens fulla längd. Eftersom jag arbetade med ett antal
  ljudfiler, bestämde jag mig för att stycket inte skulle ha någon slutdestination. Det var mer intressant att
  rotera ljudfiler efter varje genomspelning, och införa nytt material i det reda upphackade materialet. Det
  bidrog till en ständig rörelse, men också en dekonstruktion av ljudmaterialet, och jag tänkte på något jag
  läst om fjärilar när de förpuppas. Fjärilslarven löses upp av enzymer och blir till en soppa av proteiner,
  som sedan byggstenarna som bygger fjärilen.

\renewcommand{\baselinestretch}{1}
\begin{lstlisting}[style=SuperCollider-IDE, caption=Funktion för \emph{Metamorfos}]
~metamorphosis = {|larv, fjril, chunk, x = -1|
	var rand = rrand(0, larv.size);
	var i = 0;
	if (x == rand && larv[rand] == fjril[rand]) {
		thisFunction.value(larv, fjril, chunk, rand);
	} {
		for(0, chunk, {|i|
			larv.wrapPut((rand + i), fjril.wrapAt((rand + i)));
		});
	};
};
\end{lstlisting}
\renewcommand{\baselinestretch}{1.5}

\section{Diskussionsdel}
  % Skriv även om hur mitt tröskeltillstånd möjliggjorde att jag orkade lära mig saker som jag tidigare haft en
  % stress för att lära mig, de i mina ögon upphöjda verktygen som jag tidigare hade kastat mig ut för att göra
  % musik med, utan att riktigt veta hur man gjorde, och utan ett lugn kring vad jag ville göra.

  \end{document}
