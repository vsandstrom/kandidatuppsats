\documentclass{article}
\usepackage[utf8]{inputenc}

\begin{document}
% Vad jag vill prata om och varför jag tycker att det är intressant.
% Jag beskriver varför sökt mig till det här ämnet, utifrån min egen känsla av okunskap kring en viss teknik. 
% Hur jag kunde känna mig begränsad när jag inte behärskade något, och hur det kunde göda en prestationsångest 
% som jag sedan vidarebefordrade till andra. 

\title{Examensuppsats}
\author{Viktor Sandström}
\date{Februari 2022}

\subsection{Inledning}

Under åren 2020-2021 genomgick jag en lång sjukskrivning. Jag spenderade större delen av den här perioden i
mitt hem, med undantag för sjukhusbesök, ett julfirande och min trettioårsdag. Under den här sjukdomstiden
började jag vilja skriva musik, mest utav en frustration att plötsligt ha så mycket tid, det som jag tidigare
klagat på att jag inte haft, som ett skäl för att jag inte orkade göra musik. Min idé var från början att jag
fattade tycke för de märkliga, bortglömda miljöer som finns på sjukhus och vårdmottagningar, som existerade
mitt inuti en plats som ständigt var i rörelse, men som själva såg obefolkade och stillastående ut. Som att
deras funktionalitet gått förlorad. En sån plats som jag starkt reagerat på var de innergårdar, anlagda med
promenadstråk och icke förvuxna buskar och träd och med monobloc\footnote{\emph{Monobloc (chair)} [wiki],
https://en.wikipedia.org/wiki/Monobloc\_(chair) (Hämtad 06-02-22)}-stolar som nu var täckta av alger, eller 
korridorerna med den offentliga konst som var utplacerad genom sjukhuset. Dessa miljöerna fångade mitt
intresse, men de speglade även min egen tillvaro som, likt dessa platser, kändes stillastående och tidlösa.
Tanken att tonsätta miljöerna på sjukhus, blev istället ett projekt om att tonsätta och dokumentera min egen
stillastående vardag.

Det var först när jag efter hand böjrade må bättre som jag hade ork att börja arbeta med den här idén. I och med en
förbättrning av mitt mående och ett överskott av ledig tid hade jag även bestämt mig för att lära mig
kompositionsverktyget SuperCollider\footnote{\emph{SuperCollider}, https://supercollider.github.io/ (Hämtad
06-02-22)}, som jag tidigare haft svårt med. Därav föll det sig naturligt att mycket av musiken jag skrev var
skriven med detta verktyg. 

När jag skrev musiken reflekterade jag kring hur jag skrivit musik tidigare, främst under min utbildning, att
ha som utgångspunkt en främst personlig berättelse. Under min utbildning
kände jag att jag hade jagat ett ideal av vad elektroakustisk konstmusik är. En känsla av att inte hinna lära
sig alla tekniker är aktuella inom det elektroakustiska fältet, och att ständigt göra musik som en på ett
eller annat sätt är en del av en lärandeprocess, men som är påeldad av prestationsångest och en känsla av att
inte veta vad man gör. En insikt var att mycket, om inte allt jag gjort under åren hade varit något likt en
"tech-demo", ett ljudande exempel på vad denna teknik kan erbjuda, utan att kunna ha en känslomässig
förankring till materialet. 

Platser som någon gång i tiden haft ett ändamål för både patienter och personal att vistas i, kanske ett
avbrott från miljön runt omkring. Den offentliga konsten i korridorerna, eller som delade utrymme med en innergård, anlagd med buskar och träd  Dessa platser präglade denna tid, men det flöt även in i min egen tillvaro, då jag i min sjukskrivning
på något sätt stod still, likt dessa platser. Mitt projekt, som från början var en idé om att tonsätta de här
platserna, blev istället ett projekt om att tonsätta min egen stillastående vardag. 


////////

% Delalande-stycket är inte så fängslande, sätt det efter när jag skriver privat
I Delalands uppsats om lyssnandebeteenden beskriver han tre unika former av lyssnande\footnote{François
Delalande (1998) \emph{Music analysis and reception behaviours: Sommeil by Pierre Henry}, Journal of New Music
Research, 27:1-2, 13-66, DOI: 10.1080/09298219808570738}. Han beskriver dem som det "taxonomiska", det
"empatiska" och det "figurativa". Det första, det taxonomiska, är det som enligt Delalande förhåller sig
närmast det traditionellt musikteoretiska analyserandet, att klassificera musikens beståndsdelar, deras
"morfologiska enheters"\footnote{PIERRE SCHAEFFER!!!}. Det taxonomiska intresserar sig främst om relationerna
mellan olika identifierbara ljud och klanger och hur de utvecklas över tid. I kontrast till det taxonomiska,
det empatiska och figurativa skiljer sig just i sin tidsuppfattning av musiken och de tolkar inbördes
relationer av ljud. Det empatiska lyssnandet, till skillnad från taxonomiskt, intresserar sig med att
uppfatta musiken horisontellt, det omedelbara samspelet mellan klanger och hur det får en att känna. Istället
för att höra helheten och dess samspel, insatser och uttoningar, försöker man sätta sig i det emotionella
tillståndet som musiken, ögonblick för ögonblick försätter en i. 

% Börja här? beskriv det sista lyssnarbeteendet kort, sedan direkt in på min egen berättelse, för att sedan
% återknyta ordentligt med Delalande.

Det tredje och sista lyssnandebeteendet som
Delalande identifierat är det figurativa. Den som lyssnar med detta perspektiv är intresserad av att associera
fritt kring de ljud den hör, likt miljöer som beskriver ett narrativ. Där musiken översätts i lyssnarens huvud
till att besrkiva en scen eller plats som lyssnaren förnimmer i lyssnandet av musiken. 

Detta klingade väldigt rätt hos mig när jag börjat arbeta med ett projekt som till viss del handlade om en
sjukskrivning som jag genomgick under vintern och våren av 2021. Det började med att jag fattade tycke för de
märkliga bortglömda miljöer som finns på sjukhus och mottagningar, där det känns som man kan ana en
funktionalitet men som gått förlorad. Platser som någon gång i tiden haft ett ändamål att låta både patienter
och personal tänka på något annat. Den offentliga i korridorer, eller som delade utrymme med
en innergård, anlagd med buskar och träd och med monobloc\footnote{\emph{Monobloc
(chair)} [wiki], https://en.wikipedia.org/wiki/Monobloc\_(chair) (Hämtad 06-02-22)}-stolar som nu var täckta av
alger. Dessa platser präglade denna tid, men det flöt även in i min egen tillvaro, då jag i min sjukskrivning
på något sätt stod still, likt dessa platser. Mitt projekt, som från början var en idé om att tonsätta de här
platserna, blev istället ett projekt om att tonsätta min egen stillastående vardag. 

I den här processen, framförallt när jag mådde bättre, började jag arbeta med musiken, framförallt utifrån
verktyget SuperCollider\footnote{\emph{SuperCollider}, https://supercollider.github.io/ (Hämtad 06-02-22)}.
När jag skrev musiken reflekterade jag kring hur jag skrivit musik tidigare, främst under min utbildning, och
att detta utgångsläge, att gestalta något utifrån ett minne eller känsla, var annorlunda. Under min utbildning
kände jag att jag hade jagat ett ideal av vad elektroakustisk konstmusik är. En känsla av att inte hinna lära
sig alla tekniker är aktuella inom det elektroakustiska fältet, och att ständigt göra musik som en på ett
eller annat sätt är en del av en lärandeprocess, men som är påeldad av prestationsångest och en känsla av att
inte veta vad man gör. En insikt var att mycket, om inte allt jag gjort under åren hade varit något likt en
"tech-demo", ett ljudande exempel på vad denna teknik kan erbjuda, utan att kunna ha en känslomässig
förankring till materialet. 

% Vrida lite på första stycket, börja personligt, sedan skapa en ram i inledningens andra del, t ex Delalande,
% Schaeffer,  Smalley - morfologiska begrepp och teori kring kategorisering av elektroakustisk musik
% Mark Fisher - kritik mot new public management.




\subsection{Rubriker}

\subsubsection{State of the art}
  % Som state of the nation - var befinner sig det elektroakustiska fältet just nu? Citat och ev om min egen
  % musik. 


  

\subsection{Diskussionsdel}


\subsubsection{Absolute vs Programmatic music.}
- Joanna Demers - Listening through the noise: om hur man diskuterat vad fokus borde ligga i högre
  musikutbildningar, och att "absolut" musik har dominerat.

  % Russolo - fokus och intresset kring krigsljud och krigsmaskiner, något som ligger till grund för tidig eam,
  % som är manligt kodat och fascistiskt.

  Russolo och futuristerna på början av 1900talet var intersserade av maskiner, industri och krig, med tydliga
  fascistiska konnotationer. Deras arv kan fortfarande höras inom den elektroakusiska genren, med sitt
  intresse för ljud som kommit som biprodukt ur industri och från maskiner, och stundtals rent aggressiva
  ljud, som t ex noise. 

  % Kode9 och hans avhandling där han myntar begreppet "sonic warfare", en återkoppling till fascinationen
  % kring vapen, men här omvänt ur en granskande kritiserande blick från en elektronisk kompositör.

  % Emma Frid - Diverse sounds, kap 2.2 music diversity and inclusion - "Varför ska vi sträva efter diversity?"
  % s. 18, empowerment through music and technology


\subsubsection{Musik som vetenskaplig forskning}
- vad har det haft för konsekvenser för hur och vilken musik som skrivs, när materialet på utbildningar filtreras
  genom ett behov av att motivera sig som ett akademiskt ämne. Att detta fokus på att få vara vetenskaplig
  forskning har satt sitt spår i vad och hur man lär ut komposition, och främst elektronisk musik som är så
  fokuserad på teknik. 

  Mark Fishers tankar om hur utbildningar behöver motivera sin egen existens, kanske citatet om vilket
  förhållande lärare (instutition) och elev har, vem som köper en tjänst av vem. Att högre utbildningar
  köper en tjänst, behöver ha elever som läser för att få ekonomiskt stöd, istället för att eleven köper sin
  utbildning av instutitionen. 

	
  % Neo-liberalism och New Public Management

  % Edu-art - konst som är producerad inom akademin, som inte har något värde utanför den världen.

  % som konstnärer behöver man inte ge ifrån sig något som är bra.
  % som forskare så kan man inte tweaka 

  % Kölnstudion
  % Parisstudion

  % I came into the studio to make noises speak, I stumbled into music - Pierre Schaeffer

  % Henk Borg - Frailing

  % on, for or in the arts: 
	% on: musikvetenskap, musikhistoria - forska på musiken själv
	% for: barockinterpretation - syftet är att spela barock
	% in: 

	% konstnärlig forskning vad är det?

	% Henri Bergson - Introduction to metaphysics

  % Denis Smalley - om division of labor

  % Simon Emmerson - The relation of language to materials

  % Ens subjekt, ens röst, som man odlar, kontra t ex AI-musik, där du kodar bort ditt eget subjekt
  % I want to hear someones music, not someones max-patch. i AI är det kanske omvänt
  % AI är kanske den utlimata kritiken mot det jag kritiserar. 

  % Åsa Stjerna - Before Sound - om att släppa kontrollen om musiken 
  % Cybernetic manifesto

\subsubsection{Teknik som inspiration}
- Att inspireras av teknik, dess möjligheter och dess begränsningar. Detta är en viktig rubrik, då det är ett
  en viktig del av elektronisk musik, och en viktig del av mångas metod. 

  Linus Hillborgs examensarbete. exemplet med metal gear solid

\subsubsection{Tekniken som en börda}
- Arbetssätt och teknologi som en merit, där ju otillgängligare något är, eller mer obskyrt, desto större vikt
  har det inom elektronisk musikkretsar. 

  Konceptet gate-keeping, att hålla på "the fruits of your labor" istället för att dela kunskap, för att ha
  ett övertag över ens jämlikar.

  Maria Horns examensarbete - Om manligt och kvinnligt kodade egenskaper, och hur tekniken varit något som män
  länge fått höra att de har rätt att använda, för de har rätt att kontrollera världen, medan kvinnor inte har
  det. Detta har bidragit till att kvinnor inte har lika lätt in i rum som dominerats av män, men en
  tekno-jargong som passivt (eller aktivt) exkluderar kvinnor. 

  Fokuset på att bygga sitt verktyg, och att man kan gå vilse i den processen. Jean Baudrillard, högre
  upplösning av mediet, lägre upplösning av innehållet(?). 
  La Meme Young-podcasten med Jessica Ekomane, där de pratar om den hypotetiska masterstudenten som spenderar
  2 år att bygga sin enorma \emph{patch}, som ska lösa alla problem och äntligen möjliggöra den där musiken
  som hen alltid drömt om, för att sedan gå vilse i syftet och tappa siktet på slutmålet, att göra musik. 

  % t ex stycken som brukar extended techniques, men utan att veta varför
  % Buchla - "we are not limited by our tools but our mindsets"

  % vad är antitesen till "megapatchen"? att aldrig spara något? att göra något i stunden (live-kodning)? 
  % en skillnad mellan oss och andra instrumentalister.


\subsection{Koppling till musiken}
- Koppla till min musik, till EPn som jag gjorde som start för examensprojektet, och även de experiment som
  jag gjorde med ljud på sample-nivå. 
- Började med en idé om att tonsätta och dokumentera platser, de jag såg under min sjukskrivning, framförallt
  de övergivna offentliga rum som fanns i de miljöerna. Hur det projektet övergick till att dokumenter min
  egen vardag, som på något sätt länkades ihop med de platserna. Beskriv hur begreppet liminalitet kan
  användas för att beskriva båda dessa, då det beskriver att vara i tröskeln av någonting. En tröskelplats, en
  plats på vägen mellan två platser, eller en plats i en övergångsperiod, på väg till att bli något annat.
  Eller ett tröskeltillstånd, som i ritualen, där det finns ett före och ett efter, men under ritualen
  befinner man sig i ett mellanläge. Hur min situation som sjukskriven och bunden till mitt hem kunde ses som
  ett sådant tröskeltillstånd. 
  
  (Junk Space - tuffa arkitekten som Malte nämnde)

  Skriv även om hur mitt tröskeltillstånd möjliggjorde att jag orkade lära mig saker som jag tidigare haft en
  stress för att lära mig, de i mina ögon upphöjda verktygen som jag tidigare hade kastat mig ut för att göra
  musik med, utan att riktigt veta hur man gjorde, och utan ett lugn kring vad jag ville göra.










\end{document}
